% !TeX root       = solidfmm.tex
% !TeX program    = lualatex
% !TeX encoding   = utf-8
% !TeX spellcheck = en_GB
\documentclass{scrbook}
\KOMAoption{chapterprefix}{true}
\KOMAoption{numbers}{noendperiod}
\KOMAoption{chapterentrydots}{true}
\setcapindent{1em}
\addtokomafont{disposition}{\rmfamily}

\usepackage[british]{babel}
\usepackage[english=british]{csquotes}

\usepackage{amsmath}
\usepackage{amssymb}
\usepackage{mathtools}

\usepackage{fontspec}
\usepackage{unicode-math}

%%
%% Alternative font options
%%
%\setmainfont{TeX Gyre Bonum}
%\setmathfont{Tex Gyre Bonum Math}
%\setmainfont{TeX Gyre Pagella}
%\setmathfont{Tex Gyre Pagella Math}
%\setmainfont{TeX Gyre Termes}
%\setmathfont{Tex Gyre Termes Math}

\setmainfont[Numbers=OldStyle,Ligatures={TeX,Common}]{TeX Gyre Schola}
\setmathfont{Tex Gyre Schola Math}
\setsansfont[Scale=MatchLowercase]{TeX Gyre Heros}
%\setmonofont{TeX Gyre Cursor}
\setmonofont[Scale=MatchLowercase,Contextuals={Alternate}]{Fira Code}

% A bug in the math fonts: they do not have the setminus.
\AtBeginDocument{\renewcommand\setminus{\smallsetminus}}

\usepackage{xcolor}
\newcommand{\colour}{\color}
\newcommand{\colourlet}{\colorlet}
\colourlet{grey}{gray}
\colourlet{darkgrey}{darkgray}

\usepackage{hyperref}
\usepackage{hyperxmp}
\colourlet{internallinkcolour}{green!50!black}
\colourlet{externallinkcolour}{red!50!black}
\hypersetup
{
  pdfauthor          = {Matthias Kirchhart},
  pdfsubject         = {Documentation},
  pdftitle           = {solidfmm},
  pdfkeywords        = {C++, library, fast multipole method, Solid harmonics},
  pdfencoding        = {unicode},
  pdfcopyright       = {Copyright \textcopyright\ 2022 Matthias Kirchhart.
Permission is granted to copy, distribute and/or modify this document under
the terms of the ‘GNU Free Documentation License’, Version 1.3 or any
later version published by the Free Software Foundation; with no Invariant
Sections, Front-Cover Texts, and no Back-Cover Texts. A copy of this licence is
included in Appendix A of this work.},
  pdflicenseurl      = {https://www.gnu.org/licenses/fdl-1.3.html},
  pdflang            = {en-GB},
  pdfdisplaydoctitle = {true},
  unicode            = {true},
  colorlinks         = {true}, % Set to false for print version.
  allcolors          = internallinkcolour,
  urlcolor           = externallinkcolour,
  frenchlinks        = false,
  pdfborder          = {0 0 0},
  naturalnames       = false,
  hypertexnames      = false,
  breaklinks         = true
}

\XMPLangAlt{x-default}{%
pdfcopyright={Copyright \textcopyright\ 2022 Matthias Kirchhart.
Permission is granted to copy, distribute and/or modify this document under the
terms of the ‘GNU Free Documentation License’, Version 1.3 or any later version
published by the Free Software Foundation; with no Invariant Sections, Front-%
Cover Texts, and no Back-Cover Texts. A copy of this licence is included in
Appendix A of this work.}
}

\usepackage[nameinlink,noabbrev,capitalize]{cleveref}

\usepackage{listings}
\usepackage{lstfiracode}
\lstloadlanguages{bash,[11]C++}
\lstset{basicstyle=\ttfamily,style=FiraCodeStyle}
\colourlet{commentcolour}{darkgrey!50!cyan}
\lstdefinestyle{bash}
{
    language=bash,
    commentstyle=\colour{commentcolour},
    keywordstyle={}
}


\lstdefinestyle{cpp}
{
    language=[11]C++,
    commentstyle=\colour{commentcolour},
    keywordstyle=\colour{green!60!black},
    directivestyle=\colour{red!50!black},
    stringstyle=\colour{purple},
    morekeywords={size_t}
}

\usepackage{tikz}
\usetikzlibrary{math,fadings}
\usepackage[listings,skins,breakable]{tcolorbox}
\usepackage{booktabs}

\usepackage
[
	backend=biber,
	language=british,
	sorting=none,
    sortcites=true,
	sortlocale=en-GB,
    giveninits=true,
    maxnames=4,
    isbn=true,
    doi=true,
    autocite=superscript
]{biblatex}
\DeclareCiteCommand{\supercite}[\mkbibsuperscript]{%
\iffieldundef{prenote}{}{\BibliographyWarning{Ignoring prenote argument}}%
\iffieldundef{postnote}{}{}}%
{\bibopenbracket\usebibmacro{citeindex}\usebibmacro{cite}\usebibmacro{postnote}\bibclosebracket}{\supercitedelim}{}

\addbibresource{literature.bib}

\recalctypearea
\usepackage{microtype}


% Finally, our own commands.
\newcommand{\solidfmm}{\texttt{solidfmm}}
\newcommand{\PtoM}{\textsc{p2m}}
\newcommand{\PtoL}{\textsc{p2l}}
\newcommand{\MtoM}{\textsc{m2m}}
\newcommand{\MtoL}{\textsc{m2l}}
\newcommand{\LtoL}{\textsc{l2l}}
\newcommand{\naturals}{\ensuremath{\mathbb{N}}}
\newcommand{\integers}{\ensuremath{\mathbb{Z}}}
\newcommand{\rationals}{\ensuremath{\mathbb{Q}}}
\newcommand{\reals}{\ensuremath{\mathbb{R}}}
\newcommand{\complex}{\mathbb{C}}
\newcommand{\wholespace}{\ensuremath{\reals^3}}

\newcommand{\vv}[1]{\ensuremath{\symbf{#1}}} % Vectors
\newcommand{\dd}{\ensuremath{\symrm{d}}}     % Differential d
\newcommand{\bigO}[1]{\ensuremath{\symcal{O}\left({#1}\right)}}

\begin{document}

\newtcblisting[blend into=listings]{commandshell}[1]%
{
    boxsep=-4pt,
    enhanced,
    drop fuzzy shadow,
    listing remove caption=false,
	listing only,
    listing options={style=bash,basicstyle=\ttfamily\footnotesize},
	coltitle=black,
    attach boxed title to bottom left={yshift=-5pt,xshift=-4pt},
    boxed title style={enhanced jigsaw, colback=white, sharp corners,
                       boxrule=0pt,colframe=white},
    title={#1},
    breakable
}

\newtcblisting{commandshell*}%
{
    boxsep=-4pt,
    enhanced,
    drop fuzzy shadow,
	listing only,
    listing options={style=bash,basicstyle=\ttfamily\footnotesize},
    breakable,
}

\newtcblisting[blend into=listings]{cppcode}[1]{%
    boxsep=-4pt,
    enhanced,
    drop fuzzy shadow,
    listing remove caption=false,
	listing only,
    listing options={style=cpp,basicstyle=\ttfamily\footnotesize},
	coltitle=black,
    attach boxed title to bottom left={yshift=-5pt,xshift=-4pt},
    boxed title style={enhanced jigsaw, colback=white, sharp corners,
                       boxrule=0pt,colframe=white},
    title={#1},
    breakable
}

\newtcblisting{cppcode*}%
{
    boxsep=-4pt,
    enhanced,
    breakable,
    drop fuzzy shadow,
	listing only,
    listing options={style=cpp,basicstyle=\ttfamily\footnotesize},
    breakable
}

\frontmatter

\subject{Documentation}
\title{\solidfmm}
\subtitle{Version 1.2}
\author{Matthias Kirchhart}
\date{January 2022}

\uppertitleback
{%
Dr.~Matthias Kirchhart\\
Angewandte und Computergestützte Mathematik\\
Rheinisch-Westfälische Technische Hochschule Aachen\\
Schinkelstraße 2\\
52062 Aachen\\
Germany\\[\baselineskip]
\begin{tabular}{@{}ll}
E-mail:  &
\texttt{%
\href{mailto:kirchhart@acom.rwth-aachen.de}{kirchhart@acom.rwth-aachen.de}
}\\
Website: & 
\url{http://www.acom.rwth-aachen.de/5people/kirchhart/start}
\end{tabular}
}

\lowertitleback
{%
Copyright © 2022 Matthias Kirchhart.\\[1em]
Permission is granted to copy, distribute and/or modify this document under the
terms of the \enquote{GNU Free Documentation License}, Version 1.3 or any later
version published by the Free Software Foundation; with no Invariant Sections,
no Front-Cover Texts, and no Back-Cover Texts. A copy of this licence is
included in \Cref{chp:gnufdl} of this work.
}


\maketitle

\tableofcontents

\mainmatter

\part{User Guide}

\chapter{Quick Start}\label{chp:quickstart}
This chapter is written for people who already have an understanding of 
the fast multipole method and the expansions it uses. The definition of these
expansions alongside with the notational conventions employed by \solidfmm\ are
given \cref{chp:mathbackground}. That chapter also contains an introduction to
these topics. A complete reference of the user interface is given
in~\cref{chp:userreference}.

We assume that you are familiar with the command line and a POSIX shell, that
you have some background in programming in C++, and that you know how to use
the standard development toolchain on GNU/Linux systems.

\section{Licencing and Copyright}
I, Matthias Kirchhart, am the sole author and copyright holder of \solidfmm.
\solidfmm\ is free software. You may use, modify, and redistribute \solidfmm\
under the terms and conditions of the \enquote{GNU General Public License} (GPL)
as published by the Free Software Foundation, either version three, or, at your
option, any later version. You should have received a copy of this licence
document together with \solidfmm, it can also be found in \cref{chp:gnugpl} of
this book. 

The GPL gives you a lot of freedoms. However, it is a copyleft licence.
Consider you write your own piece of software that makes use of \solidfmm. Then,
in case you decide to publish or redistribute your software, you \emph{must}
distribute it under the terms of the GPL. In particular, the GPL \emph{does not}
allow you to incorporate \solidfmm\ into closed-source, non-free, commercial
software for retail. If you wish to do so, please contact me to negotiate
alternative licencing options.

\section{Installing \solidfmm}

\subsection{System Requirements}
To compile and install \solidfmm\ you will need a fairly recent version of the
GNU Compiler Collection (\lstinline|g++|) \emph{or} LLVM (\lstinline|clang++|)
with support for C++17. 

Apart from this, on Unix systems you will only need standard development tools,
i.\,e., a POSIX compatible shell (like \lstinline|bash|), and \lstinline|make|.
There are  no external library dependencies except for the standard C++ library.
It should also be possible to compile under Windows using Cygwin or MinGW, but I
have not tested this. If you downloaded \solidfmm\ from GitHub, you will
additionally need the GNU Autotools: \lstinline|autoconf|, \lstinline|automake|,
\lstinline|libtool|, and the software they depend on.

The reason for requiring specific compilers is the following. The performance
critical parts of the code---the so-called microkernels---are written in
assembly language, in the form of \emph{inline assembly} which is directly
embedded in the C++ code. Unfortunately, there is no consensus on how to do
this, and different compilers have different formats. Even worse, Microsoft's
compiler (MSVC) simply does not support inline assembly at all anymore.%
\footnote{We quote: \enquote{Inline assembly is not supported on the ARM and x64
processors.} \url{https://docs.microsoft.com/en-us/cpp/assembler/inline/%
inline-assembler?view=msvc-170}} \solidfmm\ uses GCC's syntax, because GCC is
available for almost all platforms, it is free software, and fairly up-to-date
with current C++ standards. LLVM also uses the same syntax. Additionally, the
code uses GCC's feature \lstinline|__builtin_cpu_supports()| to detect the
availability of AVX and AVX-512 at runtime. This extension is also supported by
LLVM. 


\subsection{Download Sources}
\solidfmm\ is available from two alternative sources:
\begin{enumerate}
\item\emph{Recommended:} From my institution's cloud server at
\url{https://rwth-aachen.sciebo.de/s/YIJFvSERVBiOkbc}.
\item From GitHub at \url{https://github.com/vorticle/solidfmm}. This repository
mostly exists to increase visibility and for the convenience of people who like
GitHub. I do not use GitHub for development, as I host my repositories myself.
Only the current releases will be pushed to this repository, without the
development history or anything else. Please \emph{do not} send me pull requests
or similar things via GitHub but contact me directly via E-mail. You will
additionally need the GNU Autotools to use these releases.
\end{enumerate}

In this text we will focus on the first option. The cloud server directory
contains all releases of \solidfmm\ as compressed \emph{tarballs}, which have
file endings \lstinline|.tar.xz|. Additionally, for each release, there is a
digital signature with ending \lstinline|.tar.xz.asc|. By default, you would
choose the latest release and download both the tarball and the respective
digital signature file. We will assume that you created a folder on your machine
for these files and opened a shell there. If you download \solidfmm\ for the
first time, you should also download the file 
\lstinline|kirchhart_pubkey.gpg|, which contains my OpenPGP public key.

\subsection{Verification (Optional)}
The next step is to verify the authenticity of the downloaded files. For this
you will need the GNU Privacy Guard, also known as GnuPG.\autocite{gnupg} Its
executable is commonly called \lstinline|gpg|. It comes preinstalled by default
on most current GNU/Linux distributions. Here we will only describe how to use
this software to verify the authenticity of \solidfmm, a detailed user guide
of \lstinline|gpg| is given on its project website. We will only describe the
process for the tarball release, which is the recommended download option.
For the GitHub release we assume that you have sufficient familiarity with
\lstinline|git| to perform the verification yourself.


\subsubsection{Setting up GnuPG}
GnuPG only needs to be set up once. If you have already done so during
installations of previous versions, you can safely skip this section.

You will need your own keypair. If you do not know what this is, you probably
do not have one yet. In this case you can easily create one by entering
\begin{commandshell*}
gpg --gen-key
\end{commandshell*}
and following the instructions on your screen. This a local procedure: no data
is submitted to the internet and your privacy is preserved.

Once you have your own keypair, you can import my key into GnuPG.
\begin{commandshell*}
gpg --import kirchhart_pubkey.gpg
\end{commandshell*}
You now need to double check if my key itself has not been compromised. This
can \emph{only} be achieved by \emph{manually} comparing it to a second,
trustworthy source. To do this, first display the key's fingerprint:
\begin{commandshell*}
gpg --fingerprint "Matthias Kirchhart"
\end{commandshell*}
This command should show something like:
\begin{commandshell*}
pub   rsa4096 2012-11-02 [SC]
      BC55 D501 A8DC 2D9B D2C9  95F3 00D1 76A4 818C 2BCE
uid        [ ultimativ ] Matthias Kirchhart <matthias.kirchhart@rwth-aachen.de>
uid        [ ultimativ ] Matthias Kirchhart <kirchhart@acom.rwth-aachen.de>
sub   rsa4096 2012-11-02 [A]
sub   rsa4096 2012-11-02 [E]
\end{commandshell*}
The long string of characters \enquote{BC55 D501 A8DC…} is my public key's
fingerprint. You have to compare this fingerprint with the one given on
my personal website on
\url{http://www.acom.rwth-aachen.de/5people/kirchhart/start}.
\emph{Only if the two fingerprints match \textbf{completely}}, you have
successfully imported my public key. If they do not match, something strange
is going on and you should abort immediately, check your computer for viruses,
etc.

Usually, however, things should work out just fine. In this case, you sign this
process off by adding your signature:
\begin{commandshell*}
gpg --sign-key "Matthias Kirchhart"
\end{commandshell*}
This procedure might seem rather tedious, but security comes at a price. Also
note  that these steps only need to be carried out once. You will never need to
do this again, unless you move to a different computer and forget moving your
GnuPG data. You do \emph{not} need to set up GnuPG again when installing
updates to \solidfmm.

\subsubsection{Verifying the Tarball}
Once you set up GnuPG, verifying the tarball is easy. Simply run the following
command in the folder into which you downloaded the tarball and the signature
file. 
\begin{commandshell*}
gpg --verify solidfmm-1.2.tar.xz.asc solidfmm-1.2.tar.xz
\end{commandshell*}
The output should contain a message stating that the signature is correct. If
so, verification is complete. Otherwise you have downloaded compromised files
and should abort \emph{immediately}. Replace the version number {1{.}2} with the
version you downloaded in the above command, if necessary.

\subsection{Unpacking the Tarball}
Once you have downloaded and (optionally) verified the tarball, it can be
unpacked as follows.
\begin{commandshell*}
tar Jxf solidfmm-1.2.tar.xz
\end{commandshell*}
For this command to work, you will need \lstinline|tar|\autocite{gnutar} and the
XZ utils\autocite{xzutils}, which come preinstalled by default on most current
GNU/Linux systems. This should create a folder named \lstinline|solidfmm-1.2|,
containing the source code of \solidfmm. Enter this folder by typing
\begin{commandshell*}
cd solidfmm-1.2
\end{commandshell*}

\subsection{Preparing a GitHub Release}
If you downloaded \solidfmm\ from GitHub instead, you will need the GNU
Autotools to generate the build system and the necessary scripts.
\begin{commandshell*}
autoreconf --install
\end{commandshell*}
After this compilation and installation instructions are identical to those
for the tarball releases.

\subsection{Configuring and Compiling}
\solidfmm\ uses the GNU Autotools to configure, build, and install the library.
These tools give you great flexibility in doing this and are well beyond the
scope of this manual. These tools are, for example, described in detail in John
Calcote's book\autocite{calcote2010}. In the simplest setting you can just
enter the source directory and configure and compile using:
\begin{commandshell*}
./configure && make
\end{commandshell*}

The \lstinline|configure| script allows you to control many aspects of the
process. For example, it is possible to only build the static library, or to
only compile the dynamically linked version. You can chose the compiler,
linker, optimisation flags, installation directories, and many more using
environment  variables and options to the script. It is even possible to
cross-compile for a different system. If you want to install \solidfmm\ in
a non-standard location, you can do so by passing the \lstinline|--prefix|
option:
\begin{commandshell*}
./configure --prefix=/your/preferred/location && make
\end{commandshell*}
However, you should only use this option if you know what you are doing: if you
install \solidfmm\ to a non-standard location, you will need to configure your
compiler and linker such that they know where to look for \solidfmm\ when using
the library. A quick overview over some of the options that
\lstinline|configure| supports can be seen by executing the following command:
\begin{commandshell*}
./configure --help
\end{commandshell*}
\noindent More detailed options are given in the file \lstinline|INSTALL|.

One important aspect for \solidfmm\ is the choice of compiler. Only recent
versions of the GNU Compiler Collection, i.\,e., \lstinline|g++|,%
\autocite{gcc} and LLVM, i.\,e., \lstinline|clang++|,\autocite{clang} are
supported. To specify the compiler to use, you can use the environment variable
\lstinline|CXX|. For example:
\begin{commandshell*}
# To compile using clang++ use:
CXX=clang++ ./configure && make

# To compile using g++ use:
CXX=g++ ./configure && make
\end{commandshell*}
\noindent If you do not specify \lstinline|CXX|, your system's default compiler
will be used.

\subsection{Installing}
When compilation is complete, you can install \solidfmm. Unless you specified
a different \lstinline|--prefix| when running \lstinline|configure|, \solidfmm\
will be installed into your system's default locations for libraries and
headers. These directories are protected, and can only be written to by the
system administrator. On Unix systems you would thus first change to the
administrator's account and install afterwards:
\begin{commandshell*}
su # Changes to the system administrator's account, requires his or her password
make install
exit # Important: leave administrator's account afterwards
\end{commandshell*}

Some current GNU/Linux distributions do not allow the use of \lstinline|su|
anymore, because they claim it would be unsafe. As so often, when used properly,
this is not true; however \lstinline|su| does have a history of being used
incorrectly. For this reason these distributions promote the use of
\lstinline|sudo| instead. If the above command does not work, on these systems,
you can try using the following instead:
\begin{commandshell*}
sudo make install
\end{commandshell*}
This command requires \emph{your} password and not that of the administrator. It
will only work if the system administrator has granted you the necessary rights.


\section{Using the Library}

\subsection{Headers and Linker Flags}
The user interface consists of only four header files. Simply
\lstinline|#include| them at the beginning of your source files:
\begin{cppcode*}
#include <solidfmm/solid.hpp>
#include <solidfmm/handles.hpp>
#include <solidfmm/harmonics.hpp>
#include <solidfmm/translations.hpp>
\end{cppcode*}

All classes and functions of the library lie in the C++ namespace
\lstinline|solidfmm|. When creating your programme, you will need to link
against the library. For example, when using \lstinline|g++|, this can be
achieved by adding the flag \lstinline|-lsolidfmm| as follows:
\begin{commandshell*}
# Compile your programme
g++ -o yourprogramme.o -c yourprogramme.cpp

# Link it into an executable
g++ -o yourprogramme -lsolidfmm yourprogramme.o
\end{commandshell*}

\subsection{Initialisation}\label{sec:quickinit}
Two kinds of objects are necessary to make use of the library: an 
\lstinline|operator_handle|, which \emph{should} be shared among different
threads, and for each thread a \lstinline|buffer_handle|, which \emph{must not}
be shared among  different threads. To create these objects, you need to know
the maximum order of multipole expansions that you are going to use. For
example, if you know that you will never need expansion orders beyond 30, and
you are working on a system with 8 cores, you would do the following:
\begin{cppcode*}
using std::vector;
using solidfmm::operator_handle;
using solidfmm::buffer_handle;

// The maximum number of concurrent threads you are going to use.
const size_t nthreads { 8 };

// Only one such object is necessary; it can and should be shared among threads.
operator_handle<double> op  { 30 }; 
                                     
// Each thread needs its own buffer; it must not be shared.
vector<buffer_handle<double>> buffers(nthreads,op.make_buffer());
\end{cppcode*}
If you want to use single precision numbers, you can replace 
\lstinline[style=cpp]|double| with \lstinline[style=cpp]|float|. For single
precision the maximum order is limited by 18, due to the limits of this
format. For double precision, it is safe to pass numbers larger than 30 to
\lstinline|operator_handle|'s constructor, however, in this case, it would come
with the cost of a slightly increased memory footprint.

\subsection{Creating Expansions}
Suppose we were given $N$ point charges at locations $(x_i,y_i,z_i)^\top
\in\wholespace$ with associated charges $m_i\in\reals$, $i=0,\dotsc,N-1$. We
now seek to create a multipole expansion of order $P=\text{15}$ around the point
$(x_A,y_A,z_A)^\top\in\wholespace$. To do this so-called \textsc{p2m} operation
in
\solidfmm, one could use the following code.
\begin{cppcode*}
using solidfmm::solid;
using solidfmm::fmadd;
using solidfmm::harmonics::R; // Regular harmonics.

size_t P { 15 };             // The order of the expansion you want to use.
size_t N;                    // Number of point charges, initialised somewhere else.
const double *x, *y, *z, *m; // Coordinates and charges, initialised somewhere else.

// Objects of type solid store coefficients of multipole or local expansions.
// Pass the desired order of the expansion. Initialised with zero coefficients.
// You can also use "float" for single precision.
solid<double> M( P ); 

// Add the contribution of each point mass to multipole expansion.
for ( size_t i = 0; i < N; ++i )
    fmadd( m[i], R<double>(P,x[i]-xA,y[i]-yA,z[i]-zA), M );
\end{cppcode*}

After this, the object \lstinline|M| of type \lstinline|solid<double>| contains
the coefficients $M_n^m\in\complex, n=0,\dotsc,P-1,\ m=-n,\dotsc,n$ of the
multipole expansion.

Although much less common, local expansions can be created directly from the
given point charges (\textsc{p2l}) in almost the exact same way. All we need to
do is to replace the regular harmonics with the singular ones. Thus, assume we
wanted to create a local expansion for the same charges around the centre
$(x_B,y_B,z_B)^\top\in\wholespace$. This can be achieved as follows:
\begin{cppcode*}
using solidfmm::solid;
using solidfmm::fmadd;
using solidfmm::harmonics::S; // Singular harmonics.

solid<double> L( P ); 
for ( size_t i = 0; i < N; ++i )
    fmadd( m[i], S<double>(P,x[i]-xB,y[i]-yB,z[i]-zB), L );
\end{cppcode*}

We note that \solidfmm\ also supports vector-valued expansions. For this we
refer the reader to \cref{chp:userreference}.

\subsection{Evaluating Expansions}
Evaluation of multipole and local expansions can be done using \lstinline|dot|.
Assume we were given the coefficients $M_n^m\in\complex$, $n=0,\dotsc,P-1$,
$m=-n,\dotsc,n$ of a multipole expansion of order $P$ around the centre
$(x_A,y_A,z_A)^\top\in\wholespace$. To evaluate this expansion at some
point $(x,y,z)^\top\in\wholespace$, you would do the following:
\begin{cppcode*}
using solidfmm::dot;
using solidfmm::solid;
using solidfmm::harmonics::S; // Singular harmonics.

solid<double> M;    // Given coefficients,     initialised somewhere else.
double xA, yA, zA;  // Given expansion centre, initialised somewhere else.
double x,  y,  z;   // Given evaluation point, initialised somewhere else.

double result { 0 };
dot( M, S<double>(P,x-xA,y-yA,z-zA), &result );
\end{cppcode*}

Local expansions work similarly: here you need to replace the singular
harmonics with the regular ones. Thus, assuming you were given the
coefficients $L_n^m\in\complex$, $n=0,\dotsc,P-1$, $m=-n,\dotsc,n$ of a
local expansion at centre $(x_B,y_B,z_B)^\top\in\wholespace$, you would
evaluate it as follows:
\begin{cppcode*}
using solidfmm::dot;
using solidfmm::solid;
using solidfmm::harmonics::R; // Regular harmonics.

solid<double> L;    // Given coefficients,     initialised somewhere else.
double xB, yB, zB;  // Given expansion centre, initialised somewhere else.
double x,  y,  z;   // Given evaluation point, initialised somewhere else.

double result { 0 };
dot( M, R<double>(P,x-xB,y-yB,z-zB), &result );
\end{cppcode*}

Again, we remark that \solidfmm\ also supports vector-valued expansions and
refer to \cref{chp:userreference} for this.

\subsection{Translations}\label{sec:quicktranslations}
We now come to the main reason for the existence of this library: to provide an
efficient implementation of the translation operators \textsc{m2m},
\textsc{m2l}, and $\textsc{l2l}$. Assume we were given the following data,
initialised
somewhere else:
\begin{cppcode*}
// Number of translations you want to carry out.
size_t howmany;

// Arrays of size howmany, containing *pointers* to coefficients.
solidfmm::solid<double> *input [howmany];   
solidfmm::solid<double> *output[howmany];  

// Shift vectors for the translations. That is, if the input coefficients
// represent expansions about a point (xA,yA,zA), and the output coefficients
// should be around another centre (xB,yB,zB), the shift-vectors would be
// (xB-xA,yB-yA,zB-zA).
double xshift[howmany]; 
double yshift[howmany];
double zshift[howmany];
\end{cppcode*}

We also assume that you already have initialised the library, as described
in \cref{sec:quickinit}. To perform the actual translation, you just need to
call \textsc{m2l}, or, respectively \textsc{m2m} and \textsc{l2l}, as follows:
\begin{cppcode*}
solidfmm::m2l(op,buffers[threadno],howmany,input,output,xshift,yshift,zshift);
\end{cppcode*}
Here, \lstinline|threadno| contains the number of the currently running
thread, such that the operation uses the corresponding buffer. \lstinline|m2m|
and \lstinline|l2l| are called analogously.

The result of \lstinline|input[i]| is \emph{added} to \lstinline|output[i]|.
Suppose you have 200 input expansions that you want to accumulate into a single
output expansion \lstinline|L|. To achieve this, you can just repeat the
output pointer:
\begin{cppcode*}
for ( int i = 0; i < 200; ++i )
	output[i] = &L;

solidfmm::m2l(op,buffers[threadno],howmany,input,output,xshift,yshift,zshift);
\end{cppcode*}
Similarly, input pointers can also be repeated if you want to translate one
source to many targets. To make maximum use of the vectorisation, always call
these operations with \lstinline|howmany| as big as possible. That is,
\emph{do not} call \lstinline|m2l| 200 times with \lstinline|howmany = 1|.

You may call \lstinline|m2m|, \lstinline|l2l|, and \lstinline|m2l|
simultaneously from several threads. In this case, each thread needs its own
buffer. You cannot have different threads  write to the same output, this will
cause a race condition and undefined results.

Note that these operations \emph{support mixed orders}, i.\,e.,
\lstinline|input[i]| and \lstinline|output[i]| may have different expansion
orders. Make sure that the output expansions are initialised and have the
correct order! Default constructed objects of type \lstinline|solid<double>|
will have order zero, so then nothing happens. If in doubt, before doing the
translation, you may call:
\begin{cppcode*}
for ( size_t i = 0; i < howmany; ++i )
{
    output[i]->resize( P ); // Choose your desired output order here.
    output[i]->zeros();     // Set all coefficients to zero.
}
\end{cppcode*}
to set the order of the output expansions to \lstinline|P|.


\chapter{Mathematical Background}\label{chp:mathbackground}
In this chapter we will give a description of some of the mathematical
background of the fast multipole method (FMM) and its use of the solid
harmonics. If you are already familiar with these topics, this chapter will
probably contain nothing new for you, except for maybe sign and notational
conventions.

\section{Motivation: Movement of Planets}
Consider we have a set of $N$ planets in space, having location
$\vv{x}_i = (x_i,y_i,z_i)^\top\in\wholespace$ (measured in metres) and mass
$m_i\in\reals$ (measured in kilogrammes), $i=1,\dotsc,N-1$. Throughout this
text, we will use a \textbf{bold} font to indicate three-dimensional vectors.
These planets mutually attract each other through gravity. Thus, let us define
the gravimetric potential, following Newton's law of gravity:
\begin{equation}\label{eqn:gravimetric-potential}
\varphi(x,y,z)\coloneqq
G\sum_{i=0}^{N-1}\frac{m_i}{\sqrt{(x-x_i)^2 + 
(y-y_i)^2 + (z-z_i)^2}} = 
G\sum_{i=0}^{N-1}\frac{m_i}{|\vv{x}-\vv{x}_i|},
\qquad
\end{equation}
where $|\vv{x}-\vv{x}_i|$ denotes the Euclidean distance between the point
$\vv{x}=(x,y,z)^\top\in\wholespace$ and the location $\vv{x}_i$ of planet $i$,
and $G\approx6.674\,\times\,10^{-11}\frac{\symrm{m}^3}{\symrm{kg}\,\symrm{s}^2}$
is the universal gravitational constant. Then, applying Newton's laws of
motion, the planet movements obey the following set of ordinary differential
equations (ODEs):
\begin{equation}
\frac{\dd^2\vv{x}_i}{\dd t^2} = -\nabla\varphi(\vv{x}_i) =
\sum_{\substack{j=0\\j\neq i}}^{N-1}
Gm_j\,\frac{\phantom{|}\vv{x}_j-\vv{x}_i\phantom{|^3}}{|\vv{x}_j-\vv{x}_i|^3},
\qquad i = 0,\dotsc,N-1.
\end{equation}

If you now consider entire galaxies, the number of planets is large, say
$N\approx 1\,000\,000$. In order to solve this set of ODEs, you need to
repeatedly evaluate $\nabla\varphi(\vv{x})$ at all planet locations
$\vv{x}_i$. Each evaluation costs \bigO{N}, giving you a total cost of
$\bigO{N^2}=\bigO{10^{12}}$. This cost is prohibitive, even for large super
computers. The fast multipole method (FMM) allows you to evaluate
\emph{approximations} of $\varphi(\vv{x})$ and its gradient and reduces this
cost from \bigO{N^2} to \bigO{N}. This results in dramatic speed ups, and it 
is this speed up alone that makes large scale simulations of galaxies possible.

\section{Expansions}
\subsection{Expansion of the Kernel Function}
Let us denote by $k(\vv{x},\vv{y})\coloneqq |\vv{x}-\vv{y}|^{-1}$ the so-called
\emph{kernel function}. Using this function, the gravimetric potential from
equation~\eqref{eqn:gravimetric-potential} can more compactly be given as
\begin{equation}\label{eqn:kgravimetric-potential}
\varphi(\vv{x})=G\sum_{i=0}^{N-1}m_i k(\vv{x},\vv{x}_i).
\end{equation}

The key to fast multipole methods is that this kernel function can be
\emph{expanded} in terms of the \emph{solid harmonics} as follows:
\begin{equation}\label{eqn:kernel-expansion}
k(\vv{x},\vv{y}) \approx
\sum_{n=0}^{P-1}\sum_{m=-n\vphantom{0}}^n
S_n^m(\vv{x})\overline{R_n^m(\vv{y})}
\qquad\text{ whenever }|\vv{x}| > |\vv{y}|.
\end{equation}
The number $P\in\naturals$ is called the \emph{order} of the expansion.
The functions $S_n^m:\wholespace \setminus \lbrace\vv{0}\rbrace\to\complex$ and
$R_n^m:\wholespace\to\complex$ are respectively called the \emph{singular} and
\emph{regular solid harmonics}. While these functions are complex-valued---the
overline on $R_n^m$ denotes complex conjugation---the expansion always takes
real values. We will define these functions precisely in following section. For
now it suffices to remember that this expansion is very rapidly converging, it
satisfies the following error-bound:\autocite[Theorem~3.2]{greengard1997}
\begin{equation}
\left|\,
k(\vv{x},\vv{y}) - 
\sum_{n=0}^{P-1}\sum_{m=-n\vphantom{0}}^n
S_n^m(\vv{x})\overline{R_n^m(\vv{y})}
\,\right|
\leq \frac{1}{|\vv{x}|-|\vv{y}|}\left(\frac{|\vv{y}|}{|\vv{x}|}\right)^P.
\end{equation}
In other words, its accuracy exponentially increases with order $P$,
additionally the approximation gets more accurate as $|\vv{x}|$ increases and
$|\vv{y}|$ decreases.

\subsection{Multipole Expansions (\PtoM)}
We now again consider the gravimetric potential $\varphi$ as given in equation~%
\eqref{eqn:kgravimetric-potential}. Furthermore assume the case where planets
are clustered around some point $\vv{x}_A\in\wholespace$, and we would like to
evaluate $\varphi$ at locations $\vv{x}$ that are far away from that cluster,
i.\,e., $|\vv{x}-\vv{x}_A| > |\vv{x}_i-\vv{x}_A|$ for all $i=0,\dotsc,N-1$. We
then have, by means of the kernel expansion~\eqref{eqn:kernel-expansion}:
\begin{multline}
\varphi(\vv{x})
= \sum_{i=0}^{N-1}Gm_i k(\vv{x},\vv{x}_i)
= \sum_{i=0}^{N-1}Gm_i k(\vv{x}-\vv{x}_A,\vv{x}_i-\vv{x}_A) \\
\stackrel{\eqref{eqn:kernel-expansion}}{\approx}
\sum_{n=0}^{P-1}\sum_{m=-n\vphantom{0}}^n
S_n^m(\vv{x}-\vv{x}_A)
\overline{\quad\underbrace{\sum_{i=0}^{N-1}R_n^m(\vv{x}_i-\vv{x}_A)Gm_i}_%
{\eqqcolon M_n^m}\quad}.
\end{multline}

This suggests the following approach:
\begin{enumerate}
\item Compute the \emph{multipoles} $M_n^m\in\complex$:
\begin{equation}\label{eqn:multipole-coefficients}
M_n^m \coloneqq \sum_{i=1}^N Gm_iR_n^m(\vv{x}_i-\vv{x}_A).
\end{equation}
Cost: $\mathcal{O}(NP^2)$.
\item Evaluate the \emph{multipole expansion} (or, if requested, its gradient)
at the desired locations:
\begin{equation}\label{eqn:multipole-expansion}
\varphi(\vv{x})\approx
\sum_{n=0}^{P-1}\sum_{m=-n\vphantom{0}}^n \overline{M_n^m}S_n^m(\vv{x}-\vv{x}_A)
\qquad
|\vv{x}-\vv{x}_A| > |\vv{x}_i-\vv{x}_A|, i=1\dotsc,N-1.
\end{equation}
Cost: \bigO{P^2} per evaluation point $\vv{x}$.
\end{enumerate}

If we only want to evaluate $\varphi$ at a single location $\vv{x}$, then
nothing is gained. However, suppose you want to evaluate $\varphi$ at
$M$ locations, where both $N,M\approx 1\,000\,000$. Then the cost for direct
evaluation is $\bigO{NM}=\bigO{10^{12}}$, whereas the approach via the
multipole expansion only costs $\bigO{NP^2 + MP^2}=\bigO{10^6P^2}$. This
results in the aforementioned dramatic speed up, assuming that $P$ is not too
large. In practice, choices up to $P=20$ are common---remember that the kernel
expansion~\eqref{eqn:kernel-expansion} converges exponentially with $P$. Only
very few calculations require even higher accuracies. Operation
\eqref{eqn:multipole-coefficients} computes multipole coefficients $M_n^m$ from
a given set of planets or particles; it is commonly called \textsc{p2m}:
particle-to-multipole.

In summary, a multipole expansion approximates the potential $\varphi$ of a
particle cluster around some centre $\vv{x}_A$ \emph{outside of this cluster}.
The expansion gets \emph{more accurate as the distance increases.} This is
illustrated in \cref{fig:multipole-expansion}.

\begin{figure}
\centering
\begin{tikzpicture}
\fill [fill=white!30!black]
(0,0) circle [radius=1.5cm];

\draw (0,0) node [anchor=north west] {$\vv{x}_A$};
\draw[mark=x,mark size=4,only marks] plot coordinates { (0,0) };

\shade [shading=radial, inner color=black, outer color=white,even odd rule]
  (0,0) circle [radius=3cm]
  (0,0) circle [radius=1.5cm];

\draw [very thick] (0,0) circle [radius=1.5cm];

% Draw random planets
\clip (0,0) circle [radius=1.5cm];
\draw plot [mark=*,mark size=0.5,only marks,samples=50]
( {random(-150,150)/100}, {random(-150,150)/100} );
\end{tikzpicture}
\caption{\label{fig:multipole-expansion}Multipole expansions converge outside
of the smallest ball around their centre that contains all particles. Here, the
cluster of planets near the centre $\vv{x}_A$ is contained within the drawn
circle. Outside this region the expansion converges and gets more accurate as
the distance increases, indicated by grey fading into white. The accuracy
increases faster for expansions of higher order $P$. The expansion does not
converge within the circle and gives large errors there, indicated by a dark
grey colour.}
\end{figure}

\begin{figure}
\centering
\begin{tikzpicture}
\fill [fill=white!30!black]
  (-4,-3) rectangle (4,3) 
  (0,0) circle [radius=2cm];

\tikzfading[name=fade in,  inner color=transparent!100,
                           outer color=transparent!0]
\tikzfading[name=fade out, inner color=transparent!0,
                           outer color=transparent!100]
%\shade [outer color=white!50!black, inner color=white, path fading=fade in]
%  (0,0) circle [radius=2cm];
\fill [white!20!black, path fading=fade in]
  (0,0) circle [radius=2cm];
\fill [white, path fading=fade out]
  (0,0) circle [radius=2cm];
\fill [white, path fading=fade out]
  (0,0) circle [radius=2cm];
\fill [white, path fading=fade out]
  (0,0) circle [radius=2cm];


\draw plot [mark=x,only marks,mark size=4] (0,0);
\draw (0,0) node [anchor=north west] {$\vv{x}_B$};
\draw [very thick] (0,0) circle [radius=2cm];

\clip (-4,-3) rectangle (4,3)
      (0,0) circle [radius=2cm];

\draw plot [mark=*,mark size=0.5,only marks,samples=500]
( {random(-400,400)/100}, {random(-300,300)/100} );

\end{tikzpicture}
\caption{\label{fig:local-expansion}Local expansions converge in the largest
ball around their centre that contains no particles. The error of a local
expansion is zero at its centre~$\vv{x}_B$, indicated by a shade of white in
this figure. It then gradually increases when approaching the boundary of the
ball (white fading to grey). This increase is slower for expansions of higher
order $P$. Outside of this ball, the local expansion does not converge and
gives large errors (dark grey).}
\end{figure}

\subsection{Local Expansions (\PtoL)}
Local expansions in some ways consider the opposite case, as illustrated in
\cref{fig:local-expansion}. Here we want to evaluate $\varphi$ at a cluster of
locations $\vv{x}$ around some centre $\vv{x}_B$, where the particles lie
\emph{outside} of this cluster. In other words, we have $|\vv{x}-\vv{x}_B| <
|\vv{x}_i-\vv{x}_B|$ for all $i=1,\dotsc,N-1$, and thus:
\begin{multline}
\varphi(\vv{x})
= \sum_{i=0}^{N-1}Gm_i k(\vv{x},\vv{x}_i)
= \sum_{i=0}^{N-1}Gm_i k(\vv{x}_i-\vv{x}_B,\vv{x}-\vv{x}_B) \\
\stackrel{\eqref{eqn:kernel-expansion}}{\approx}
\sum_{n=0}^{P-1}\sum_{m=-n\vphantom{0}}^n
\overline{R_n^m(\vv{x}-\vv{x}_B)}
\underbrace{\sum_{i=0}^{N-1}S_n^m(\vv{x}_i-\vv{x}_B)Gm_i}_{\eqqcolon L_n^m}.
\end{multline}

After computing the \emph{local coefficients} $L_n^m\in\complex$, the potential
$\varphi$ can be approximated near $\vv{x}_B$ using the local expansion:
\begin{equation}
\varphi(\vv{x})\approx\sum_{n=0}^{P-1}\sum_{m=-n\vphantom{0}}^{n}
L_n^m\overline{R_n^m(\vv{x}-\vv{x}_B)}\qquad
|\vv{x}-\vv{x}_B| < |\vv{x}_i-\vv{x}_B|,\ i=1,\dotsc,N-1.
\end{equation}

The operation of computing the local coefficients directly from the particles
is called particle-to-local (\textsc{p2l}):
\begin{equation}\label{eqn:p2l}
L_n^m\coloneqq\sum_{i=0}^{N-1}S_n^m(\vv{x}_i-\vv{x}_B)Gm_i.
\end{equation}
However, most FMM implementations do not compute local coefficients via
\textsc{p2l}; those that do only use it in rare circumstances. The usual way to
obtain local coefficients is the so-called \textsc{m2l} translation and will be
discussed later.

\subsection{Remarks on the Derivation of Expansions}
Key to all fast multipole methods is the availability of rapidly converging
series expansions of the underlying kernel function $k(\vv{x},\vv{y})$, such
as in \cref{eqn:kernel-expansion}. Generic expansions, such as Taylor series,
can in principle be applied to any kind of kernel $k$. For example,
the Taylor expansion of order $P$ of some kernel $k(\vv{x},\vv{y})$ for a fixed
$\vv{x}$ around $\vv{y}=\vv{0}$ reads:
\begin{equation}\label{eqn:taylor-expansion}
k(\vv{x},\vv{y})\approx \sum_{|\vv{\alpha}|<P}\ 
\underbrace{\partial_{\vv{y}}^{\vv{\alpha}}k(\vv{x},\vv{0})
\vphantom{\frac{a}{b}}}_{\eqqcolon%
\widetilde{S}_{\vv{\alpha}}(\vv{x})}
\underbrace{\quad\frac{\vv{y}^{\vv{\alpha}}}{\vv{\alpha}!}\quad}_{\eqqcolon%
\widetilde{R}_{\vv{\alpha}}(\vv{y})},
\end{equation}
where $\vv{\alpha}\in\naturals^3_0$ is a multi-index. The key point is that
this expansion is a sum of products, where one factor only depends on
$\vv{x}$ and the other factor only depends on $\vv{y}$. In this way the
simultaneous dependence of $k$ on both $\vv{x}$ and $\vv{y}$ \emph{decouples}.

The Taylor expansion consists of \bigO{P^3} terms, where the functions
$\widetilde{S}_{\vv{\alpha}}$ and $\widetilde{R}_{\vv{\alpha}}$ take similar
roles as the $S_n^m$ and $R_n^m$ in \cref{eqn:kernel-expansion}. Thus, assuming
sufficient regularity of $k$, Taylor's theorem is all we need to to construct
multipole and local expansions and it also provides us with error bounds.
Without additional knowledge about $k$, this is essentially already the best we
can do. However, also note that the Taylor expansion has \bigO{P^3} terms,
while \cref{eqn:kernel-expansion} requires only \bigO{P^2}. Why is this so?

With the particular choice $k(\vv{x},\vv{y})=|\vv{x}-\vv{y}|^{-1}$ and fixed
$\vv{x}$, the Taylor polynomials~\eqref{eqn:taylor-expansion} are always
\emph{harmonic}. In other words, the exact kernel function satisfies
$-\Delta_{\vv{y}}k(\vv{x},\vv{y})=0$, where $-\Delta_{\vv{y}}$ is the Laplace
operator with respect to the $\vv{y}$-variable. This property is inherited by
the Taylor polynomials. There are only \bigO{P^2} linearly independent harmonic
polynomials, so it suffices to use a basis for this \emph{sub}space. The regular
solid harmonics $R_n^m$, $|m|\leq n$, are exactly that: they are a basis of the
space of homogeneous harmonic polynomials of degree $n$. 
Expansion~\eqref{eqn:kernel-expansion} achieves better compression because only
basis functions for the proper subspace are used. The Taylor expansion~%
\eqref{eqn:taylor-expansion}, on the other hand, uses the generic monomial
basis for \emph{all} polynomials, not just the harmonic ones.

Only for the particular kernel $k(\vv{x},\vv{y})=|\vv{x}-\vv{y}|^{-1}$ the
solid harmonics are the correct choice. \solidfmm\ only considers this
kernel function and the solid harmonics. For other kernel functions one can
either use the more expensive, but more general Taylor
expansion~\eqref{eqn:taylor-expansion}, or other generic $\bigO{P^3}$ approaches
like Chebyshev interpolation. Alternatively, if performance is critical, it
might be worthwhile to develop more efficient expansions specific to the
particular kernel at hand.

\section{Solid Harmonics}\label{sec:solidharmonics}
The solid harmonics can be motivated and derived in many different ways. We
refer the interested reader to the literature for this. Often they are given
in spherical coordinates. When restricting the solid harmonics to the surface
of the unit sphere, one obtains the more well-known \emph{spherical} harmonics
$Y_n^m(\theta,\varphi)$, i.\,e., $Y_n^m(\theta,\varphi)=
R_n^m(r=1,\theta,\varphi)$. This explains their name: unlike the spherical
harmonics $Y_n^m$, the solid harmonics $S_n^m$ and $R_n^m$ are defined not only
on the sphere's surface, but also on the volume it encloses, i.\,e., on the
entire solid sphere. They are called \emph{harmonic} because
$-\Delta S_n^m(\vv{x}) = 0$ for
$\vv{x}\in\wholespace\setminus\lbrace\vv{0}\rbrace$ and
$-\Delta R_n^m(\vv{x}) = 0$ for $\vv{x}\in\wholespace$. The functions $S_n^m$
are called singular solid harmonics because of their singularities at
$\vv{x}=\vv{0}$. The functions $R_n^m$ are polynomials, hence they are called
regular solid harmonics.

While spherical coordinates have certain benefits for analysis, these functions
can more easily evaluated in Cartesian coordinates. There are countless
conventions concerning signs and normalisations. \solidfmm\ follows the
convention given by Dehnen.~\autocite{dehnen2014} For computational purposes,
the solid harmonics are best defined recursively. Let
$\vv{x}=(x,y,z)^\top\in\wholespace$ be given, and let
$|\vv{x}|=\sqrt{x^2+y^2+z^2}$ denote its Euclidean length. Starting
with $S_0^0(\vv{x})=|\vv{x}|^{-1}$ and $R_0^0(\vv{x})=1$, one continues with
the diagonal $n=m$:
\begin{equation}
S_n^n = (2n-1)\frac{x+Iy}{|\vv{x}|^2}S_{n-1}^{n-1},\qquad
R_n^n =       \frac{x+Iy}{2n}R_{n-1}^{n-1},
\end{equation}
where $I$ denotes the imaginary unit $I^2=-1$. For the remaining non-negative
$m=0,\dotsc,n-1$, they can be computed via:
\begin{equation}
\begin{split}
|\vv{x}|^2 S_n^m    &= (2n-1)zS_{n-1}^m-\bigl((n-1)^2-m^2\bigr)S_{n-2}^m,\\
(n^2-m^2)R_n^m &= (2n-1)zR_{n-1}^m - |\vv{x}|^2R_{n-2}^m.
\end{split}
\end{equation}
Here, it is implicitly assumed that $R_{n-2}^m=S_{n-2}^m=0$ whenever $m=n-1$.

For the negative values of $m$, i.\,e., $m=-n,\dotsc,-1$, they are defined
\emph{implicitly} via
\begin{equation}\label{eqn:negativem}
S_n^m(\vv{x}) = (-1)^m\overline{S_n^{-m}(\vv{x})}
\quad\text{and}\quad
R_n^m(\vv{x}) = (-1)^m\overline{R_n^{-m}(\vv{x})}.
\end{equation}
This property therefore also holds for the coefficients of the multipole and
local expansions:
\begin{equation}\label{eqn:negativemcoeffs}
M_n^m = (-1)^m\overline{M_n^{-m}}
\quad\text{and}\quad
L_n^m = (-1)^m\overline{L_n^{-m}}.
\end{equation}

Thus, to even further increase compression, in a computer implementation
neither the solid harmonics $R_n^m$ and $S_n^m$, nor the multipole and local
coefficients $L_n^m$ and $M_n^m$ should be explicitly stored for negative $m$.
Taking these symmetries into account, a local or multipole expansion
of order $P$ consists of $P(P+1)/2$ complex numbers, which can be arranged
in a triangular pattern as shown in \Cref{fig:triangle}.

\begin{figure}
\centering
\begin{tikzpicture}
\draw[->] (-0.25,4.5) -- (-0.25,0.5);
\draw     (-0.5,2.5) node {$n$};
\draw[->] (0.5,5.25) -- (4.5,5.25);
\draw     (2.5,5.5) node {$m$};
\foreach \n in {0,1,2,3,4}%
{
    \foreach \m in {0,...,\n}%
    {
        \draw[thick]  (\m,4-\n)   rectangle (\m+1,5-\n);
        \draw (\m+0.5,4.5-\n) node [black,scale=0.9] {\Large$C_{\n}^{\m}$};
    }
}
\end{tikzpicture}
\caption{\label{fig:triangle}The coefficients $C_n^m\in\complex$,
$C\in\lbrace S, R, M, L\rbrace$ of both multipole and local expansions, as well
as the values of the solid harmonics $S_n^m$ and $R_n^m$ can be stored using
$P(P+1)/2$ complex numbers. In this example we have $P=5$. They can be arranged
in a triangular pattern according to their indices $n=0,\dotsc,4$ and
$m=0,\dotsc,n$, where the values for negative $m$ are given implicitly by
$C_n^{m}=(-1)^m\overline{C_n^{-m}}$.}
\end{figure}


At the same time, this property guarantees that the
expansion~\eqref{eqn:kernel-expansion} always is real-valued: for $m=0$ we have
$R_n^0\in\reals$ and $S_n^0\in\reals$. For $m\neq 0$ the imaginary parts for
$\pm m$ cancel each other out:
\begin{multline}
\sum_{n=0}^{P-1}\sum_{m=-n\vphantom{0}}^n S_n^m\overline{R_n^m} 
=
\sum_{n=0}^{P-1}S_n^0 R_n^0 +
\sum_{n=0}^{P-1}\sum_{m=1\vphantom{0}}^n \biggl(S_n^m\overline{R_n^m}+
S_n^{-m}\overline{R_n^{-m}}\biggr)\\
\stackrel{\eqref{eqn:negativem}}{=}
\sum_{n=0}^{P-1}\underbrace{S_n^0R_n^0}_{\in\reals} +
\sum_{n=0}^{P-1}\sum_{m=1\vphantom{0}}^n
\underbrace{\biggl(S_n^m\overline{R_n^m}+
\overline{S_n^m}R_n^m\biggr)}_{\in\reals}.
\end{multline}



\section{Translations}
Translation operators take in one expansion around centre $\vv{x}_A$ and produce
another around some centre $\vv{x}_B$. When both in- and output expansions are
of the same order $P$, the translation operators map \bigO{P^2} input 
coefficients to \bigO{P^2} output coefficients at the cost of \bigO{P^4}
arithmetic operations. The so-called \MtoL-translation is particularly
important for FMMs, and a significant portion of the computational time is
spent on performing these translations. They are the main reason for the
existence of this library: it provides highly efficient, vectorised 
implementations of the translation operators. Additionally, \solidfmm's
implementation is based on the approach outlined by
Dehnen,\autocite{dehnen2014} which reduces the complexity from \bigO{P^4} to
\bigO{P^3} in theory and even \bigO{P^2} in practice.

\subsection{Multipole-to-Multipole (\MtoM)}
\MtoM\ may be interpreted just like \PtoM, with the difference that the input
is not a planet with mass $m_i$ at location $\vv{x}_i$, but a multipole
expansion of order $P_i$ with coefficients $M_{n,i}^m$ and centre
$\vv{x}_i$. Thus, in \MtoM, both the input and output are multipole expansions.
We again denote by $\vv{x}_A$ and $P$ the centre and order of the output
expansion. We furthermore introduce the \emph{shift vector} $\vv{r}=
\vv{x}_A-\vv{x}_i$, pointing from the old to the new expansion centre.

Instead of expanding the kernel $k(\vv{x},\vv{y})=S_0^0(\vv{x}-\vv{y})$, we
now expand the singular solid harmonics $S_n^m(\vv{x}-\vv{y})$, analogously to
expansion~\eqref{eqn:kernel-expansion}:\autocite[Equation~(49)]{dehnen2014}
\begin{equation}\label{eqn:Sexpansion}
S_n^m(\vv{x}-\vv{y}) = \sum_{k=0}^{\infty}\sum_{l=-k\vphantom{0}}^{k}
S_{n+k}^{m+l}(\vv{x})\overline{R_k^l(\vv{y})}\qquad |\vv{x}| > |\vv{y}|.
\end{equation}
Thus, whenever $|\vv{x}-\vv{x}_A| > |\vv{x}_i-\vv{x}_A|$, the input multipole
expansion with centre $\vv{x}_i$ can itself be expanded around $\vv{x}_A$ as
follows:
\begin{multline}
\sum_{n=0}^{P_i-1}\sum_{m=-n\vphantom{0}}^{n}
\overline{M_{n,i}^m}S_n^m(\vv{x}-\vv{x}_i) \\
=
\sum_{n=0}^{P_i-1}\sum_{m=-n\vphantom{0}}^{n}
\overline{M_{n,i}^m}S_n^m\bigl((\vv{x}-\vv{x}_A)-(\vv{x}_i-\vv{x}_A)\bigr) \\
\stackrel{\eqref{eqn:Sexpansion}}{=}
\sum_{n=0}^{P_i-1}\sum_{m=-n\vphantom{0}}^{n}
\sum_{k=0}^{\infty}\sum_{l=-k\vphantom{0}}^{k}
\overline{M_{n,i}^mR_k^l(-\vv{r})}S_{n+k}^{m+l}(\vv{x}-\vv{x}_A).
\end{multline}
All that remains is a change of indices and rearranging the sums. To simplify
the summation limits, we will implicitly assume that we have $R_{n-k}^{m-l}
(-\vv{r})=0$ whenever $|m-l| > n-k$ or $n-k<0$. This finally results in:
\begin{multline}
\sum_{n=0}^{P_i-1}\sum_{m=-n\vphantom{0}}^{n}
\overline{M_{n,i}^m}S_n^m(\vv{x}-\vv{x}_i) \\
=
\sum_{n=0}^{\infty}\sum_{m=-n\vphantom{0}}^{n}
S_{n}^{m}(\vv{x}-\vv{x}_A)
\left(
\sum_{k=0}^{P_i-1}\sum_{l=-k}^{k}
\overline{M_{k,i}^{l}R_{n-k}^{m-l}(-\vv{r})}
\right).
\end{multline}
We may thus define the new multipoles $M_n^m$ via:
\begin{equation}\label{eqn:fullm2l}
M_n^m \coloneqq \sum_{k=0}^{P_i-1} \sum_{l=-n}^{n}
M_{k,i}^lR_{n-k}^{m-l}(-\vv{r}).
\end{equation}
This is the general \MtoM-translation formula. We then have, after truncating
at the given output order $P$:
\begin{equation}
\sum_{n=0}^{P_i-1}\sum_{m=-n\vphantom{0}}^{n}
\overline{M_{n,i}^m}S_n^m(\vv{x}-\vv{x}_i) 
\approx
\sum_{n=0}^{P-1}\sum_{m=-n\vphantom{0}}^{n}
\overline{M_n^m}S_n^m(\vv{x}-\vv{x}_A)
\qquad |\vv{x}-\vv{x}_A| > |\vv{x}_i-\vv{x}_A|.
\end{equation}

While \PtoM\ allows us to combine several \enquote{particles} or
\enquote{planets} into a single multipole expansion with centre $\vv{x}_A$,
\MtoM\ allows us to combine several multipole expansions into one. Suppose we
were given $N$ multipole expansions with centres $\vv{x}_i$ and orders $P_i$,
$i=0,\dotsc,N-1$. The potential approximated by these expansions:
\begin{equation}\label{eqn:m2m-originalphi}
\varphi(\vv{x})\approx
\sum_{i=0}^{N-1}\sum_{n=0}^{P_i-1}\sum_{\vphantom{0}m=-n}^{n}
\overline{M_{n,i}^m}S_n^m(\vv{x}-\vv{x}_i)
\end{equation}
can then in turn be approximated as follows:
\begin{enumerate}
\item Compute the combined multipole moments around $\vv{x}_A$ using
\MtoM:
\begin{equation}
M_n^m 
\stackrel{\eqref{eqn:fullm2l}}{=}
\sum_{i=0}^{N-1}\sum_{k=0}^{P_i-1}\sum_{l=-k}^{k}
M_{k,i}^lR_{n-k}^{m-l}(\vv{x}_i-\vv{x}_A).
\end{equation}
Cost, assuming $P_i=P$ for all $i=0,\dotsc,N-1$: \bigO{NP^4}. \solidfmm\ reduces
this cost to \bigO{NP^3} for large $P$. For all practically relevant choices
of $P$, however, the empirically measured timings behave even better, namely
optimally as~\bigO{NP^2}.
\item Evaluate the new expansion at the desired output locations:
\begin{equation}
\varphi(\vv{x}) \approx \sum_{n=0}^{P-1}\sum_{m=-n\vphantom{0}}^n
\overline{M_n^m}S_n^m(\vv{x}-\vv{x}_A).
\end{equation}
Cost: \bigO{P^2} per evaluation point $\vv{x}$.
\end{enumerate}

The resulting approximation now contains \emph{two sources of error}:
\begin{enumerate}
\item The error of the original approximation~\eqref{eqn:m2m-originalphi}.
\item The error from the \MtoM\ translation.
\end{enumerate}
The additional error behaves just like the error for an individual expansion 
obtained via \PtoM. In particular, \cref{fig:multipole-expansion} also
describes this additional error and area of convergence, where the
\enquote{dots} in this figure now correspond to the multipole expansions
$M_{n,i}^m$ and their centres $\vv{x}_i$.


\subsection{Multipole-to-Local (\MtoL)}
\MtoL\ is to \PtoL\ what \MtoM\ is to \PtoM: we can again treat a given
multipole expansion just like a planet or particle. While \PtoL\ allows
us to locally approximate the potential induced by several planets, \MtoL\
allows us to approximate the potential given by multipole expansions. Thus,
suppose we were given a multipole expansion with coefficients $M_{n,i}^m$ and
order $P_i$ around some centre $\vv{x}_i$. We now wish to locally approximate
this expansion near $\vv{x}_B$ using a local expansion. Again, we denote
by $\vv{r}=\vv{x}_B-\vv{x}_i$ the \emph{shift vector}, pointing from the old to
the new expansion centre.

We will make use of the fact that we can flip signs via $S_n^m(-\vv{x}) =
(-1)^nS_n^m(\vv{x})$. For $|\vv{x}-\vv{x}_B| < |\vv{x}-\vv{x}_i|$ we then have:
\begin{multline}
\sum_{n=0}^{P_i-1}\sum_{m=-n\vphantom{0}}^n%
\overline{M_{n,i}^m}S_n^m(\vv{x}-\vv{x}_i) \\
= 
\sum_{n=0}^{P_i-1}\sum_{m=-n\vphantom{0}}^n%
\overline{M_{n,i}^m}S_n^m\left((\vv{x}-\vv{x}_B)-(\vv{x}_i-\vv{x}_B)\right) \\
\stackrel{\text{sign flip}}{=}
\sum_{n=0}^{P_i-1}\sum_{m=-n\vphantom{0}}^n(-1)^n
\overline{M_{n,i}^m}S_n^m\left((\vv{x}_i-\vv{x}_B)-(\vv{x}-\vv{x}_B)\right) \\
\stackrel{\eqref{eqn:Sexpansion}}{=}
\sum_{n=0}^{P_i-1}\sum_{m=-n\vphantom{0}}^{n}
\sum_{k=0}^{\infty}\sum_{l=-k\vphantom{0}}^{k}(-1)^n
\overline{M_{n,i}^m}S_{n+k}^{m+l}(-\vv{r})\overline{R_k^l(\vv{x}-\vv{x}_B)} \\
\stackrel{\text{sign flip}}{=}
\sum_{n=0}^{P_i-1}\sum_{m=-n\vphantom{0}}^{n}
\sum_{k=0}^{\infty}\sum_{l=-k\vphantom{0}}^{k}(-1)^k
\overline{M_{n,i}^m}S_{n+k}^{m+l}(\vv{r})\overline{R_k^l(\vv{x}-\vv{x}_B)}.
\end{multline}
It again remains to rearrange the sums and change the indices to obtain:
\begin{multline}
\sum_{n=0}^{P_i-1}\sum_{m=-n\vphantom{0}}^n%
\overline{M_{n,i}^m}S_n^m(\vv{x}-\vv{x}_i) \\
=
\sum_{n=0}^\infty\sum_{m=-n\vphantom{0}}^n
\overline{R_n^m(\vv{x}-\vv{x}_B)}
\left(
(-1)^n
\sum_{k=0}^{P_i-1}\sum_{l=-k\vphantom{0}}^{k}
\overline{M_{k,i}^l}S_{n+k}^{m+l}(\vv{r})
\right).
\end{multline}
Thus, the general \MtoL-translation formula is as follows:
\begin{equation}
L_n^m\coloneqq (-1)^n
\sum_{k=0}^{P_i-1}\sum_{l=-k\vphantom{0}}^{k}
\overline{M_{k,i}^l}S_{n+k}^{m+l}(\vv{r}).
\end{equation}
After truncating at the output order $P$, the resulting local expansion then
approximates the original multipole expansion near $\vv{x}_B$:
\begin{equation}
\sum_{n=0}^{P_i-1}\sum_{\vphantom{0}m=-n}^n
\overline{M_n^m}S_n^m(\vv{x}-\vv{x}_i)
\approx
\sum_{n=0}^{P-1}\sum_{\vphantom{0}m=-n}^n
L_n^m\overline{R_n^m(\vv{x}-\vv{x}_B)}
\qquad |\vv{x}-\vv{x}_B| < |\vv{x}_i-\vv{x}_B|.
\end{equation}

The remarks on the error of \MtoM\ analogously carry over to \MtoL. Several
multipole expansions can be combined into a single local expansion. There
are then two sources of error: the error from the original expansions plus
the error from \MtoL. The second error contribution behaves just like in
\cref{fig:local-expansion}: it is zero at $\vv{x}_B$, and then gradually
increases as one approaches the perimeter of the largest sphere around
$\vv{x}_B$ that contains none of the $\vv{x}_i$.

\subsection{Local-to-Local (\LtoL)}
\LtoL\ differs from the previous translation operators in that it does not make
sense to choose the output's order $P$ higher than that of the input expansion~%
$P_i$. In \LtoL, the input is a local expansion of order $P_i$, i.\,e., a
harmonic polynomial of total degree less than $P_i$. The space of harmonic
polynomials is invariant under translation, i.\,e., polynomials can be
translated exactly. Thus, when $P=P_i$, \LtoL\ is an error-free operation.
Errors are only introduced when $P<P_i$, in which case they are zero at the new
expansion centre $\vv{x}_B$ and gradually increase from there.

The derivation is similar to that of \MtoM\ and \MtoL. We begin by expanding
the regular solid harmonics:\autocite[Equation~(48)]{dehnen2014}
\begin{equation}\label{eqn:Rexpansion}
R_n^m(\vv{x}-\vv{y}) =
\sum_{k=0}^n\sum_{l=-k}^k R_{k}^{l}(\vv{x})R_{n-k}^{m-l}(-\vv{y})
\qquad
(\text{exact for any $\vv{x},\vv{y}\in\wholespace$}).
\end{equation}
Here, to simplify notation, we again implicitly assume that $R_n^m=0$ whenever
$|m| > n$.

Thus, for a given input expansion $L_{n,i}^m$ around $\vv{x}_i$ and of order
$P_i$, and shift vector $\vv{r}=\vv{x}_B-\vv{x}_i$, one obtains:
\begin{multline}
\sum_{n=0}^{P_i-1}\sum_{m=-n\vphantom{0}}^n
L_{n,i}^m\overline{R_n^m(\vv{x}-\vv{x}_i)}
=
\sum_{n=0}^{P_i-1}\sum_{m=-n\vphantom{0}}^n
L_{n,i}^m\overline{R_n^m\bigl((\vv{x}-\vv{x}_B)-(\vv{x}_i-\vv{x}_B)\bigr)} \\
\stackrel{\eqref{eqn:Rexpansion}}{=}
\sum_{n=0}^{P_i-1}\sum_{m=-n\vphantom{0}}^n\sum_{k=0}^n\sum_{l=-k}^kL_{n,i}^m
\overline{R_{k}^{l}(\vv{x}-\vv{x}_B)R_{n-k}^{m-l}(\vv{r})} \\
\stackrel{\text{rearranging}}{=}
\sum_{n=0}^{P_i-1}\sum_{m=-n\vphantom{0}}^n \overline{R_n^m(\vv{x}-\vv{x}_B)}
\left(
\sum_{k=n}^{P_i-1}\sum_{l=-k}^k
L_{k,i}^l\overline{R_{k-n}^{l-m}(\vv{r})}
\right).
\end{multline}
Thus, we may define new local moments via:
\begin{equation}
L_n^m =\sum_{k=n}^{P_i-1}\sum_{l=-k}^k L_{k,i}^l\overline{R_{k-n}^{l-m}(\vv{r})}
\end{equation}
resulting in
\begin{equation}
\sum_{n=0}^{P_i-1}\sum_{m=-n\vphantom{0}}^n
L_{n,i}^m\overline{R_n^m(\vv{x}-\vv{x}_i)}
=
\sum_{n=0}^{P_i-1}\sum_{m=-n\vphantom{0}}^n
L_{n}^m\overline{R_n^m(\vv{x}-\vv{x}_B)}
\qquad\text{exactly}.
\end{equation}
The output expansion may be truncated ($P<P_i$). Choosing $(P>P_i)$ will only
result in adding zero coefficients $L_n^m=0$ for $n\geq P_i$.

\subsection{Summary of the Translation Formulæ}\label{sec:translationsummary}
All of the translation operators take an input expansion of order $P_i$ around
some centre $\vv{x}_i$ and produce a new expansion of order $P$ around centre
$\vv{x}_i+\vv{r}$, where $\vv{r}$ is the so-called \emph{shift vector}.
Depending on the type of translation, the output expansion may coincide with
the input expansion, or just be an approximation of it. The formulæ for
computing the output coefficients are as follows.
\begin{itemize}
\item[\MtoM:]
\begin{equation}
M_n^m = \sum_{k=0}^{P_i-1} \sum_{l=-n}^{n}
M_{k,i}^lR_{n-k}^{m-l}(-\vv{r}).
\end{equation}
\item[\MtoL:]
\begin{equation}
L_n^m = (-1)^n
\sum_{k=0}^{P_i-1}\sum_{l=-k}^{k}
\overline{M_{k,i}^l}S_{n+k}^{m+l}(\vv{r}).
\end{equation}
\item[\LtoL:]
\begin{equation}
L_n^m = \sum_{k=n}^{P_i-1}\sum_{l=-k}^k
L_{k,i}^l\overline{R_{k-n}^{l-m}(\vv{r})}.
\end{equation}
\end{itemize}
In these sums it is implicitly assumed that $R_n^m=0$ whenever $n<0$ or
$|m| > n$. When in- and output order $P=P_i$ coincide, computing all output
coefficients using these forumulæ costs \bigO{P^4}. \solidfmm\ uses acceleration
techniques to reduce this cost to \bigO{P^3} in theory, but measurements have
shown that in practice it performs even better as \bigO{P^2}.

\section{Outlook}
The concepts described in this chapter form the main building blocks of
fast multipole methods. With the mathematical background presented here, it
should be possible to understand the library functionality described in
\cref{chp:quickstart}. At the moment, \solidfmm\ does not provide a full
implementation of the fast multipole method, but only an efficient, optimised
implementation of these building blocks. This is intentional, so that the
library remains slim can be used in existing implementations of the FMM. For
this reason we do not give a full description of the algorithm, of which there
exist many variations.

The original FMM is due to Greengard and Rokhlin.\autocite{greengard1987}
Since then the algorithm has developed into more simple, adaptive,
parallelisable variants. A description of the \enquote{dual tree traversal}
by Dehnen\autocite{dehnen2002} and a comparison to other variants was given by
Yokota.\autocite{yokota2013b} We find this version of the FMM particularly
intuitive. He also compares different series expansions and mentions the lack
of efficient implementations for the solid harmonics, even though they are
the most efficient expansions available for the important kernel function
$k(\vv{x},\vv{y})=|\vv{x}-\vv{y}|^{-1}$. \solidfmm's aim is to provide such an
implementation.



\chapter{User Reference}\label{chp:userreference}
This chapter considers \solidfmm's API for users. The implementation details
are covered in \cref{part:developer} of this book. All components of the library
are in the \solidfmm\ namespace. For brevity, we will omit the explicitly
stating namespace in the discussion that follows. For example we will simply
write \lstinline|solid| instead of \lstinline|solidfmm::solid|.

\section{Overview}
\solidfmm\ exposes its interface through four header files, each of which
will be described in a separate section below:
\begin{itemize}
\item \lstinline|solidfmm/solid.hpp| Contains the \lstinline|solid| data
structure for storing the coefficients of expansions and values of the
solid harmonics. Also contains the \lstinline|dot| and \lstinline|fmadd|
operations.
\item \lstinline|solidfmm/harmonics.hpp| Evaluation of the solid harmonics
and their gradients.
\item \lstinline|solidfmm/handles.hpp| Handle classes for creating buffers
and shared data which are needed for the translation operations.
\item \lstinline|solidfmm/translations.hpp| The actual translation operations.
\end{itemize}

Most operations are templates and take a type parameter which is commonly
called \lstinline|real|. This type specifies which fundamental arithmetic
type should be used to represent a real number. \solidfmm\ supports single and
double precision, i.\,e., \lstinline|float| and \lstinline|double| for this
parameter. It is illegal to use other types for \lstinline|real|.

\section{\texttt{solidfmm/solid.hpp}}
Objects of type \lstinline|solid| are used to store the coefficients of both
local and multipole expansions. Additionally, when evaluating the solid
harmonics $S_n^m(\vv{x})$ and $R_n^m(\vv{x})$ at some given point
$\vv{x}\in\wholespace$, the resulting values are also stored in objects of this
type. Instances can either hold scalar or vector-valued data. For example, the
solid harmonics $S_n^m(\vv{x}), R_n^m(\vv{x})\in\complex$ are scalars. Their
gradients are vector-valued: $\nabla S_n^m(\vv{x}), \nabla
R_n^m(\vv{x})\in\complex^3$. For these reason most operations additionally take
a parameter called \lstinline|dim|, which, however, can be omitted when working
with scalars. Similarly, it is possible to use vector-valued multipole and local
expansions.

\begin{cppcode}{Public interface of the \lstinline|solid| class.}
template <typename real>
class solid
{
public:
    solid() = default;
    solid( size_t P, size_t dim = 1 );
    solid( const solid  &rhs );
    solid(       solid &&rhs ) noexcept;
    solid& operator=( const solid  &rhs );
    solid& operator=(       solid &&rhs ) noexcept;
   ~solid();

    void resize( size_t P, size_t dim = 1 );
    void reinit( size_t P, size_t dim = 1 );
    void zeros() noexcept;

    const real& re( size_t n, size_t m, size_t d = 0 ) const noexcept;
          real& re( size_t n, size_t m, size_t d = 0 )       noexcept;
    const real& im( size_t n, size_t m, size_t d = 0 ) const noexcept;
          real& im( size_t n, size_t m, size_t d = 0 )       noexcept;

    size_t order    () const noexcept;
    size_t dimension() const noexcept;

    const real*  memptr() const noexcept;
          real*  memptr()       noexcept;

private:
	// …
};
\end{cppcode}

\subsection{Construction and Resizing}
The default constructor creates instances of order $P=0$, i.\,e., empty objects
which do not contain any coefficients or values. When creating a fresh object
using the second constructor, or when calling \lstinline|reinit| with the
desired order and dimension, one obtains a \lstinline|solidfmm::solid|
containing only zero values. Calling the member \lstinline|resize|, on the
other hand, will result in object of the desired order and dimension, but with
\emph{undefined} contents. \lstinline|zeros| can be used to set all members to
zero. Thus, \lstinline|reinit| is equivalent to calling \lstinline|resize| and
\lstinline|zeros| immediately afterwards. The copy and move constructors, as
well as the assignment operators have the usual semantics. All these operations
either come with a no-throw guarantee, or the strong exception guarantee.
In particular, functions like \lstinline|resize| can only throw exceptions of
type \lstinline|std::bad_alloc|, but in case they do, they leave the object
unchanged.

\subsection{Accessing Elements}
The contents of a \lstinline|solid| can be accessed using the member functions
\lstinline|re| and \lstinline|im|, which respectively return the real and
imaginary parts of the specified element.
\begin{cppcode*}
solid<double> M; // Given multipole coefficients, initialised somewhere else.

double a = M.re(4,2);
M.im(2,1) = 42;
\end{cppcode*}
Under the assumption that \lstinline|M.order() > 4|, this code stores the value
of $\Re M_4^2$ in the variable \lstinline|a| and assigns the value~42 to
$\Im M_2^1$. For vector-valued a \lstinline|solid|, the above code would access
the zeroth component of the respective vectors; dimensions are counted
beginning with zero. For example, to access the \enquote{y-component} of the
real part of some $\vv{M}_4^2\in\complex^3$, one would instead use:
\begin{cppcode*}
double a = M.re(4,2,1);
\end{cppcode*}

The indices $n$, $m$, and $d$ (default value $d=0$) must be valid. No range
checks are performed; passing invalid arguments will result in undefined
behaviour. As discussed in \cref{sec:solidharmonics}, it is not necessary to
store the values of $M_n^m$ for the negative values of $m$, as they are given
implicitly. For this reason, for a given \lstinline|solid| of order $P$ and
dimension $\mathrm{D}$, the valid range of parameters is $0\leq n < P$,
$0\leq m\leq n$, and $0\leq d < \mathrm{D}$.

\subsection{Direct Memory Access}
The member function \lstinline|memptr()| gives direct memory access to the
data that a \lstinline|solid| is holding. The indexing scheme is best
illustrated graphically, see \cref{fig:solidmemorylayout}. Assume we were given
a \lstinline|solid| of order $P$ and dimension $\mathrm{D}$. Then the real
and imaginary parts of the $d$th vector-component of some
$\vv{M}_n^m\in\complex^{\mathrm{D}}$, $0\leq n < P$, $0\leq m\leq n$,
$0\leq d<\mathrm{D}$ can be accessed via:
\begin{cppcode*}
double real_part = M.memptr()[ d*P*(P+1) + (n  )*(n+1) + m ];
double imag_part = M.memptr()[ d*P*(P+1) + (n+1)*(n+1) + m ];
\end{cppcode*}

Don't even think about calling \lstinline|delete|, \lstinline|delete []|,
\lstinline|free()|, \lstinline|realloc()|, or the like on this 
pointer---unless, of course, you are keen on causing undefined behaviour and
crashing your programme.

\begin{figure}
\centering
\begin{tikzpicture}
\draw[->] (-0.25,4.5) -- (-0.25,0.5);
\draw     (-0.5,2.5) node {$n$};
\draw[->] (0.5,5.25) -- (4.5,5.25);
\draw     (2.5,5.5) node {$m$};
\foreach \n in {0,1,2,3,4}%
{
    \foreach \m in {0,...,\n}%
    {
        \fill[gray]   (\m,4-\n)   rectangle (\m+1,4.5-\n);
        \fill[white]  (\m,4.5-\n) rectangle (\m+1,5-\n);
        \draw[thick]  (\m,4-\n)   rectangle (\m+1,5-\n);
        \draw (\m+0.3,4.5-\n) node [black,scale=0.9] {\Large$C_{\n}^{\m}$};

        \tikzmath{ integer \reidx;
                   integer \imidx;
                   \reidx = (\n  ) * (\n+1) + \m;
                   \imidx = (\n+1) * (\n+1) + \m; };
        \draw (\m+0.75,4.80-\n) node [scale=0.8] {\texttt{\reidx}};
        \draw (\m+0.75,4.20-\n) node [scale=0.8] {\texttt{\imidx}};
    }
}
\draw (2,-0.5) node {$\texttt{pos}\bigr(\Re(C_n^m)\bigr)= (n+0)(n+1) + m,$};
\draw (2,-1.0) node {$\texttt{pos}\bigl(\Im(C_n^m)\bigr)\,= (n+1)(n+1) + m.$};
\draw (5.5,4.5) node[left,align=left] {Real parts\\Imaginary parts};
\filldraw[fill=white] (2,4.64) rectangle +(0.2,0.2);
\filldraw[fill=gray]  (2,4.22) rectangle +(0.2,0.2);
\end{tikzpicture}
\caption{\label{fig:solidmemorylayout}A generic \lstinline|solid| $C$ may
contain coefficients of local or multipole expansions, as well as the values of
regular or singular harmonics. The memory layout of a \emph{scalar}
\lstinline|solid| of order~5 is illustrated above. Only the values $C_n^m$ for
$m\geq 0$ need to be stored, as the values for negative $m$ are given
implicitly. The array position \lstinline|pos| of the respective values is
indicated by the numbers in the upper and lower right corners of the boxes. A
vector-valued \lstinline|solid| first stores all values for dimension $d=0$ as
shown above. Then all elements for dimension $d=1$ follow using the same
layout, then those of dimension $d=2$, and so forth.}
\end{figure}

\subsection{Evaluating Expansions using \texttt{dot}}
Consider the task of evaluating a given multipole expansion around some centre
$\vv{x}_A$:
\begin{equation}
\sum_{n=0}^{P-1}\sum_{m=-n\vphantom{0}}^n\overline{M_n^m}S_n^m(\vv{x}-\vv{x}_A)
\end{equation}
as discussed in \cref{chp:mathbackground}. This operation is essentially the
dot-product on two complex-valued vectors $M,S\in\mathbb{C}^{P^2}$ which are
indexed in a non-standard way. Additionally, as discussed before, this sum
always takes real values, so the roles of $M$ and $S$ can be swapped without
changing the result. For this reason, \solidfmm\ implements such
operations using the function \lstinline|dot|.
\begin{cppcode*}
void dot( const solid<float>  &A, const solid<float>  &B, float  *result ) noexcept;
void dot( const solid<double> &A, const solid<double> &B, double *result ) noexcept;
\end{cppcode*}

This function supports inputs of differing orders and dimensions, i.\,e.,
it is legal to have \lstinline|A.order() != B.order()| and
\lstinline|A.dimension() != B.dimension()|. The semantics of these cases are
as follows.

\begin{enumerate}
\item In case we have \lstinline|A.order() != B.order()|, the \lstinline|solid|
of lower order is padded with zeros to match the other \lstinline|solid|'s
order.
\item In case we have \lstinline|A.dimension()!=1| or
\lstinline|B.dimension()!=1|, all combinations of components are dotted
together, i.\,e., the result is a matrix of dimension
\begin{equation*}
\texttt{A.dimension()} \times \texttt{B.dimension()}
\end{equation*}
that is stored in row-major order. The entry at position $(i,j)$ then contains
the value
\begin{equation*}
\sum_{n=0}^{P-1}\sum_{m=-n\vphantom{0}}^n\overline{A_{n,i}^m}B_{n,\,j}^m.
\end{equation*}
If you prefer column-major order, you can just swap \lstinline|A| and
\lstinline|B| when calling \lstinline|dot|.
\end{enumerate}
For this reason, \lstinline|dot| does not simply return a floating point value,
but takes a pointer to the first element where the result should be stored. It
is the duty of the user to ensure that these pointers point to a region of
memory which is sufficiently large to store the result.

Assume you were given a vector-valued local expansion of order $P$ with
coefficients $\vv{L}_n^m\in\complex^3$ around expansion centre $\vv{x}_B=
(\mathtt{xB},\mathtt{yB},\mathtt{zB})^\top\in\reals^3$. To evaluate this
expansion at some location 
$\vv{x}=(\mathtt{x},\mathtt{y},\mathtt{z})^\top\in\reals^3$, you would use
the following code:
\begin{cppcode*}
double result[3];
dot( L, harmonics::R<double>(P,x-xB,y-yB,z-zB), result ); 
\end{cppcode*}
To evaluate its derivative, i.\,e., its Jacobian matrix of dimension
$\reals^{3\times 3}$, you would then use the following code and 
replace the regular harmonics \lstinline|R| with their gradients \lstinline|dR|:
\begin{cppcode*}
double jacobian[9];
dot( L, harmonics::dR<double>(P,x-xB,y-yB,z-zB), jacobian );
\end{cppcode*}
Note that calling \lstinline|harmonics::R| or \lstinline|harmonics::dR| requires
you to \lstinline[style=cpp]|#include| the \lstinline|solidfmm/harmonics.hpp|
header, which is described in \cref{sec:harmonicshpp}.

\subsection{Creating Expansions using \texttt{fmadd}}
We recall the \PtoM\ and \PtoL\ operations from \cref{chp:mathbackground}.
Given a set of \enquote{particles} with location $\vv{x}_i$ and mass $m_i$,
$i=0,\dotsc,N-1$, we desire to compute multipole or local
expansions around respective centres $\vv{x}_A$ and $\vv{x}_B$:
\begin{equation}
M_n^m = \sum_{i=0}^{N-1}m_i R_n^m(\vv{x}-\vv{x}_A),\qquad
L_n^m = \sum_{i=0}^{N-1}m_i S_n^m(\vv{x}-\vv{x}_B).
\end{equation}
This operation thus always takes multiples of a \lstinline|solid| containing the
values of the regular or singular harmonics, and accumulates them into a
\lstinline|solid| that contains the expansion coefficients. There also is
support for vector-valued expansions:
\begin{cppcode*}
void fmadd( const float   fac, const solid<float > &A, solid<float > &B );
void fmadd( const float  *fac, const solid<float > &A, solid<float > &B );
void fmadd( const double  fac, const solid<double> &A, solid<double> &B );
void fmadd( const double *fac, const solid<double> &A, solid<double> &B );
\end{cppcode*}

In all of these operations one must have both \lstinline|A.dimension() == B.dimension()| and \lstinline|A.order() == B.order()|, because it is not
clear how to meaningfully define those operations otherwise. These functions
throw exceptions of type \lstinline|std::logic_error| if one of these conditions
is violated. Again, the strong exception guarantee is given: in case an
exception is thrown, the function's arguments are not changed.

The functions taking a \emph{value} \lstinline|fac| essentially perform
\lstinline|B += fac*A|; but note that \solidfmm\ does not implement
\lstinline|operator+=|. The operations taking a \emph{pointer} \lstinline|fac|
assume that it points to an array of size \lstinline|A.dimension() == B.dimension()|; the $d$'th component is then scaled by \lstinline|fac[d]|.

We are still not sure how to devise a more flexible interface, we are open
for suggestions!


\section{\texttt{solidfmm/harmonics.hpp}}\label{sec:harmonicshpp}
This header contains the function declations for evaluating the regular and
singular harmonics, as well as their gradients, as they are defined in
\cref{sec:solidharmonics}. The interface should be pretty self-explanatory:
\begin{cppcode*}
namespace harmonics
{

template <typename real> solid<real>  R( size_t P, real x, real y, real z );
template <typename real> solid<real>  S( size_t P, real x, real y, real z );
template <typename real> solid<real> dR( size_t P, real x, real y, real z );
template <typename real> solid<real> dS( size_t P, real x, real y, real z );

}
\end{cppcode*}

The parameter \lstinline|P| specifies the order up to which you wish to evaluate
the harmonics. The parameters \lstinline|x|,\lstinline|y|, and \lstinline|z| are
the respective components of the position vector $\vv{x}=(\mathtt{x},\mathtt{y},\mathtt{z})^\top\in\wholespace$ at which
you want to evaluate the harmonic functions. Only \lstinline[style=cpp]|double|
and \lstinline[style=cpp]|float| are supported for the template parameter
\lstinline|real|. These functions may throw exceptions of type
\lstinline|std::bad_alloc|.

\section{\texttt{solidfmm/handles.hpp}}\label{sec:handles}
The handle classes encapsulate resources that are implementation details but
are needed to carry out the translation operations \MtoM, \MtoL, and \LtoL.
There are two kinds of handles:
\begin{enumerate}
\item \lstinline|operator_handle|. Only one of these handles is needed, it
can and \emph{should be shared} among threads. It is a wrapper to the an
implementation detail, which essentially stores the so-called Wigner matrices,
which are needed for the efficient implementation of the translation operators.
This class also automatically determines the available CPU features at runtime
and chooses the most efficient implementation available.
\item \lstinline|buffer_handle|. \emph{Each thread needs its own handle.} These
handles need to be created for a given \lstinline|operator_handle|. As the
name implies, a \lstinline|buffer_handle| encapsulates buffers which a thread
can use as its workspace when performing translations.
\end{enumerate}

Handles can be copied, default constructed, and also support assignment
operations. In a typical scenario, however, a user would create these handles
\emph{only once} at program start-up and then pass them around by reference.
The relevant parts of the user-interface are as follows:
\begin{cppcode*}
template <typename real>
class operator_handle
{
public:
	// …
    operator_handle( size_t P );
    buffer_handle<real> make_buffer() const;    
	// …
};

template <typename real>
class buffer_handle
{
public:
	// …
    buffer_handle( const operator_data<real> &op );
	// …
};
\end{cppcode*}

In other words, a \lstinline|buffer_handle| needs to be created specifically
for a particular \lstinline|operator_handle|. As discussed in
\cref{sec:quickinit}, this can for example be achieved as follows:
\begin{cppcode*}
// The maximum number of concurrent threads you are going to use.
// For example, on a CPU with 8 cores:
const size_t nthreads { 8 };

// Only one such object is necessary; it can and should be shared among threads.
// For exmaple, if you never need expansion orders beyond P = 30:
operator_handle<double> op  { 30 }; 
                                     
// Each thread needs its own buffer; it must not be shared.
std::vector<buffer_handle<double>> buffers(nthreads,op.make_buffer());
\end{cppcode*}

These routines may throw exceptions of type \lstinline|std::bad_alloc|. Again,
it is illegal to use other types than \lstinline[style=cpp]|double| and
\lstinline[style=cpp]|float| for the template parameter \lstinline|real|.

The translation operations make use of faculties up to $2P-2$. For this
reason, for \lstinline[style=cpp]|float| the maximum supported order is 18:
for higher orders the number $(2P-2)!$ is out of the range of this type. For
double precision calculations the corresponding limit is 86. In both cases,
the accuracies achieved by expansions of such high order are usually already
exceeding the precision of the respective floating point numbers. In this
case the constructors throw exceptions of type \lstinline|std::overflow_error|.
Passing $P=0$ will case an exception of type \lstinline|std::logic_error| to be
thrown.


\section{\texttt{solidfmm/translations.hpp}}
This header contains the actual translation operations:
\begin{cppcode*}
void m2m( const operator_data<double> &op, threadlocal_buffer<double> &buf, 
          size_t howmany, const solid<double> *const *const Min,
                                solid<double> *const *const Mout,
          const double *x, const double *y, const double *z );

void m2l( const operator_data<double> &op, threadlocal_buffer<double> &buf, 
          size_t howmany, const solid<double> *const *const M,
                                solid<double> *const *const L,
          const double *x, const double *y, const double *z );


void l2l( const operator_data<double> &op, threadlocal_buffer<double> &buf, 
          size_t howmany, const solid<double> *const *const Lin,
                                solid<double> *const *const Lout,
          const double *x, const double *y, const double *z );
\end{cppcode*}
and the corresponding functions for single precision calculations using
\lstinline[style=cpp]|float|.

The types \lstinline|operator_data| and \lstinline|threadlocal_buffer| are the
implementation details that are encapsulated by the handle classes in
\cref{sec:handles}. Thus, a user should pass the corresponding handles instead,
and \emph{never} use the internal classes \lstinline|operator_data| and
\lstinline|threadlocal_buffer| directly.

The mathematical background and details of these operations are discussed in
\cref{chp:mathbackground}. The usage of these functions is already illustrated
in \cref{sec:quicktranslations}. In addition to the scenario described there,
these operations also support vector-valued \lstinline|solid|s. In this case
the translations are carried out component-wise along the specified shift
vector. For this to work, the target \lstinline|solid|s must match the
dimension of the input expansion. In other words, for example for 
\MtoL\ one must have for all $i=0,\dotsc,\mathtt{howmany}-1$:
\begin{center}
\lstinline|M[i]->dimension() == L[i]->dimension()|.
\end{center}

Again, these functions provide the strong exception guarantee. In case an
exception is thrown, they leave their arguments unchanged. The following
exceptions can be thrown:
\begin{itemize}
\item \lstinline|std::logic_error| If input and output dimensions of the passed
\lstinline|solid|s differ. This also is thrown if you pass a
\lstinline|buffer_handle| that was created for a different
\lstinline|operator_handle|.
\item \lstinline|std::out_of_range| if the order of one of the passed
\lstinline|solid|s exceeds the order of the given \lstinline|operator_handle|.
\end{itemize}

\part{Developer Guide}\label{part:developer}

\chapter{Microkernels}
I am currently still working on a detailed developer guide. In the mean time,
for an overview of the implementation techniques used by \solidfmm, I refer to
the white paper at:
\begin{center}
\url{https://rwth-aachen.sciebo.de/s/YIJFvSERVBiOkbc}
\end{center}

\appendix
\part*{Appendices}
\addparttocentry{}{Appendices}

\chapter{GNU Free Documentation License}
\label{chp:gnufdl}

 \begin{center}

       Version 1.3, 3 November 2008


 Copyright \copyright{} 2000, 2001, 2002, 2007, 2008  Free Software Foundation, Inc.
 
 \bigskip
 
     \url{<https://fsf.org/>}
  
 \bigskip
 
 Everyone is permitted to copy and distribute verbatim copies
 of this license document, but changing it is not allowed.
\end{center}


\begin{center}
{\Large\bfseries 0. PREAMBLE\par}
\end{center}
The purpose of this License is to make a manual, textbook, or other
functional and useful document ``free'' in the sense of freedom: to
assure everyone the effective freedom to copy and redistribute it,
with or without modifying it, either commercially or noncommercially.
Secondarily, this License preserves for the author and publisher a way
to get credit for their work, while not being considered responsible
for modifications made by others.

This License is a kind of ``copyleft'', which means that derivative
works of the document must themselves be free in the same sense.  It
complements the GNU General Public License, which is a copyleft
license designed for free software.

We have designed this License in order to use it for manuals for free
software, because free software needs free documentation: a free
program should come with manuals providing the same freedoms that the
software does.  But this License is not limited to software manuals;
it can be used for any textual work, regardless of subject matter or
whether it is published as a printed book.  We recommend this License
principally for works whose purpose is instruction or reference.


\begin{center}
{\Large\bfseries 1. APPLICABILITY AND DEFINITIONS\par}
\end{center}
This License applies to any manual or other work, in any medium, that
contains a notice placed by the copyright holder saying it can be
distributed under the terms of this License.  Such a notice grants a
world-wide, royalty-free license, unlimited in duration, to use that
work under the conditions stated herein.  The ``\textbf{Document}'', below,
refers to any such manual or work.  Any member of the public is a
licensee, and is addressed as ``\textbf{you}''.  You accept the license if you
copy, modify or distribute the work in a way requiring permission
under copyright law.

A ``\textbf{Modified Version}'' of the Document means any work containing the
Document or a portion of it, either copied verbatim, or with
modifications and/or translated into another language.

A ``\textbf{Secondary Section}'' is a named appendix or a front-matter section of
the Document that deals exclusively with the relationship of the
publishers or authors of the Document to the Document's overall subject
(or to related matters) and contains nothing that could fall directly
within that overall subject.  (Thus, if the Document is in part a
textbook of mathematics, a Secondary Section may not explain any
mathematics.)  The relationship could be a matter of historical
connection with the subject or with related matters, or of legal,
commercial, philosophical, ethical or political position regarding
them.

The ``\textbf{Invariant Sections}'' are certain Secondary Sections whose titles
are designated, as being those of Invariant Sections, in the notice
that says that the Document is released under this License.  If a
section does not fit the above definition of Secondary then it is not
allowed to be designated as Invariant.  The Document may contain zero
Invariant Sections.  If the Document does not identify any Invariant
Sections then there are none.

The ``\textbf{Cover Texts}'' are certain short passages of text that are listed,
as Front-Cover Texts or Back-Cover Texts, in the notice that says that
the Document is released under this License.  A Front-Cover Text may
be at most 5 words, and a Back-Cover Text may be at most 25 words.

A ``\textbf{Transparent}'' copy of the Document means a machine-readable copy,
represented in a format whose specification is available to the
general public, that is suitable for revising the document
straightforwardly with generic text editors or (for images composed of
pixels) generic paint programs or (for drawings) some widely available
drawing editor, and that is suitable for input to text formatters or
for automatic translation to a variety of formats suitable for input
to text formatters.  A copy made in an otherwise Transparent file
format whose markup, or absence of markup, has been arranged to thwart
or discourage subsequent modification by readers is not Transparent.
An image format is not Transparent if used for any substantial amount
of text.  A copy that is not ``Transparent'' is called ``\textbf{Opaque}''.

Examples of suitable formats for Transparent copies include plain
ASCII without markup, Texinfo input format, LaTeX input format, SGML
or XML using a publicly available DTD, and standard-conforming simple
HTML, PostScript or PDF designed for human modification.  Examples of
transparent image formats include PNG, XCF and JPG.  Opaque formats
include proprietary formats that can be read and edited only by
proprietary word processors, SGML or XML for which the DTD and/or
processing tools are not generally available, and the
machine-generated HTML, PostScript or PDF produced by some word
processors for output purposes only.

The ``\textbf{Title Page}'' means, for a printed book, the title page itself,
plus such following pages as are needed to hold, legibly, the material
this License requires to appear in the title page.  For works in
formats which do not have any title page as such, ``Title Page'' means
the text near the most prominent appearance of the work's title,
preceding the beginning of the body of the text.

The ``\textbf{publisher}'' means any person or entity that distributes
copies of the Document to the public.

A section ``\textbf{Entitled XYZ}'' means a named subunit of the Document whose
title either is precisely XYZ or contains XYZ in parentheses following
text that translates XYZ in another language.  (Here XYZ stands for a
specific section name mentioned below, such as ``\textbf{Acknowledgements}'',
``\textbf{Dedications}'', ``\textbf{Endorsements}'', or ``\textbf{History}''.)  
To ``\textbf{Preserve the Title}''
of such a section when you modify the Document means that it remains a
section ``Entitled XYZ'' according to this definition.

The Document may include Warranty Disclaimers next to the notice which
states that this License applies to the Document.  These Warranty
Disclaimers are considered to be included by reference in this
License, but only as regards disclaiming warranties: any other
implication that these Warranty Disclaimers may have is void and has
no effect on the meaning of this License.


\begin{center}
{\Large\bfseries 2. VERBATIM COPYING\par}
\end{center}
You may copy and distribute the Document in any medium, either
commercially or noncommercially, provided that this License, the
copyright notices, and the license notice saying this License applies
to the Document are reproduced in all copies, and that you add no other
conditions whatsoever to those of this License.  You may not use
technical measures to obstruct or control the reading or further
copying of the copies you make or distribute.  However, you may accept
compensation in exchange for copies.  If you distribute a large enough
number of copies you must also follow the conditions in section~3.

You may also lend copies, under the same conditions stated above, and
you may publicly display copies.


\begin{center}
{\Large\bfseries 3. COPYING IN QUANTITY\par}
\end{center}
If you publish printed copies (or copies in media that commonly have
printed covers) of the Document, numbering more than 100, and the
Document's license notice requires Cover Texts, you must enclose the
copies in covers that carry, clearly and legibly, all these Cover
Texts: Front-Cover Texts on the front cover, and Back-Cover Texts on
the back cover.  Both covers must also clearly and legibly identify
you as the publisher of these copies.  The front cover must present
the full title with all words of the title equally prominent and
visible.  You may add other material on the covers in addition.
Copying with changes limited to the covers, as long as they preserve
the title of the Document and satisfy these conditions, can be treated
as verbatim copying in other respects.

If the required texts for either cover are too voluminous to fit
legibly, you should put the first ones listed (as many as fit
reasonably) on the actual cover, and continue the rest onto adjacent
pages.

If you publish or distribute Opaque copies of the Document numbering
more than 100, you must either include a machine-readable Transparent
copy along with each Opaque copy, or state in or with each Opaque copy
a computer-network location from which the general network-using
public has access to download using public-standard network protocols
a complete Transparent copy of the Document, free of added material.
If you use the latter option, you must take reasonably prudent steps,
when you begin distribution of Opaque copies in quantity, to ensure
that this Transparent copy will remain thus accessible at the stated
location until at least one year after the last time you distribute an
Opaque copy (directly or through your agents or retailers) of that
edition to the public.

It is requested, but not required, that you contact the authors of the
Document well before redistributing any large number of copies, to give
them a chance to provide you with an updated version of the Document.


\begin{center}
{\Large\bfseries 4. MODIFICATIONS\par}
\end{center}
You may copy and distribute a Modified Version of the Document under
the conditions of sections 2 and 3 above, provided that you release
the Modified Version under precisely this License, with the Modified
Version filling the role of the Document, thus licensing distribution
and modification of the Modified Version to whoever possesses a copy
of it.  In addition, you must do these things in the Modified Version:

\begin{itemize}
\item[A.] 
   Use in the Title Page (and on the covers, if any) a title distinct
   from that of the Document, and from those of previous versions
   (which should, if there were any, be listed in the History section
   of the Document).  You may use the same title as a previous version
   if the original publisher of that version gives permission.
   
\item[B.]
   List on the Title Page, as authors, one or more persons or entities
   responsible for authorship of the modifications in the Modified
   Version, together with at least five of the principal authors of the
   Document (all of its principal authors, if it has fewer than five),
   unless they release you from this requirement.
   
\item[C.]
   State on the Title page the name of the publisher of the
   Modified Version, as the publisher.
   
\item[D.]
   Preserve all the copyright notices of the Document.
   
\item[E.]
   Add an appropriate copyright notice for your modifications
   adjacent to the other copyright notices.
   
\item[F.]
   Include, immediately after the copyright notices, a license notice
   giving the public permission to use the Modified Version under the
   terms of this License, in the form shown in the Addendum below.
   
\item[G.]
   Preserve in that license notice the full lists of Invariant Sections
   and required Cover Texts given in the Document's license notice.
   
\item[H.]
   Include an unaltered copy of this License.
   
\item[I.]
   Preserve the section Entitled ``History'', Preserve its Title, and add
   to it an item stating at least the title, year, new authors, and
   publisher of the Modified Version as given on the Title Page.  If
   there is no section Entitled ``History'' in the Document, create one
   stating the title, year, authors, and publisher of the Document as
   given on its Title Page, then add an item describing the Modified
   Version as stated in the previous sentence.
   
\item[J.]
   Preserve the network location, if any, given in the Document for
   public access to a Transparent copy of the Document, and likewise
   the network locations given in the Document for previous versions
   it was based on.  These may be placed in the ``History'' section.
   You may omit a network location for a work that was published at
   least four years before the Document itself, or if the original
   publisher of the version it refers to gives permission.
   
\item[K.]
   For any section Entitled ``Acknowledgements'' or ``Dedications'',
   Preserve the Title of the section, and preserve in the section all
   the substance and tone of each of the contributor acknowledgements
   and/or dedications given therein.
   
\item[L.]
   Preserve all the Invariant Sections of the Document,
   unaltered in their text and in their titles.  Section numbers
   or the equivalent are not considered part of the section titles.
   
\item[M.]
   Delete any section Entitled ``Endorsements''.  Such a section
   may not be included in the Modified Version.
   
\item[N.]
   Do not retitle any existing section to be Entitled ``Endorsements''
   or to conflict in title with any Invariant Section.
   
\item[O.]
   Preserve any Warranty Disclaimers.
\end{itemize}

If the Modified Version includes new front-matter sections or
appendices that qualify as Secondary Sections and contain no material
copied from the Document, you may at your option designate some or all
of these sections as invariant.  To do this, add their titles to the
list of Invariant Sections in the Modified Version's license notice.
These titles must be distinct from any other section titles.

You may add a section Entitled ``Endorsements'', provided it contains
nothing but endorsements of your Modified Version by various
parties---for example, statements of peer review or that the text has
been approved by an organization as the authoritative definition of a
standard.

You may add a passage of up to five words as a Front-Cover Text, and a
passage of up to 25 words as a Back-Cover Text, to the end of the list
of Cover Texts in the Modified Version.  Only one passage of
Front-Cover Text and one of Back-Cover Text may be added by (or
through arrangements made by) any one entity.  If the Document already
includes a cover text for the same cover, previously added by you or
by arrangement made by the same entity you are acting on behalf of,
you may not add another; but you may replace the old one, on explicit
permission from the previous publisher that added the old one.

The author(s) and publisher(s) of the Document do not by this License
give permission to use their names for publicity for or to assert or
imply endorsement of any Modified Version.


\begin{center}
{\Large\bfseries 5. COMBINING DOCUMENTS\par}
\end{center}
You may combine the Document with other documents released under this
License, under the terms defined in section~4 above for modified
versions, provided that you include in the combination all of the
Invariant Sections of all of the original documents, unmodified, and
list them all as Invariant Sections of your combined work in its
license notice, and that you preserve all their Warranty Disclaimers.

The combined work need only contain one copy of this License, and
multiple identical Invariant Sections may be replaced with a single
copy.  If there are multiple Invariant Sections with the same name but
different contents, make the title of each such section unique by
adding at the end of it, in parentheses, the name of the original
author or publisher of that section if known, or else a unique number.
Make the same adjustment to the section titles in the list of
Invariant Sections in the license notice of the combined work.

In the combination, you must combine any sections Entitled ``History''
in the various original documents, forming one section Entitled
``History''; likewise combine any sections Entitled ``Acknowledgements'',
and any sections Entitled ``Dedications''.  You must delete all sections
Entitled ``Endorsements''.

\begin{center}
{\Large\bfseries 6. COLLECTIONS OF DOCUMENTS\par}
\end{center}
You may make a collection consisting of the Document and other documents
released under this License, and replace the individual copies of this
License in the various documents with a single copy that is included in
the collection, provided that you follow the rules of this License for
verbatim copying of each of the documents in all other respects.

You may extract a single document from such a collection, and distribute
it individually under this License, provided you insert a copy of this
License into the extracted document, and follow this License in all
other respects regarding verbatim copying of that document.


\begin{center}
{\Large\bfseries 7. AGGREGATION WITH INDEPENDENT WORKS\par}
\end{center}
A compilation of the Document or its derivatives with other separate
and independent documents or works, in or on a volume of a storage or
distribution medium, is called an ``aggregate'' if the copyright
resulting from the compilation is not used to limit the legal rights
of the compilation's users beyond what the individual works permit.
When the Document is included in an aggregate, this License does not
apply to the other works in the aggregate which are not themselves
derivative works of the Document.

If the Cover Text requirement of section~3 is applicable to these
copies of the Document, then if the Document is less than one half of
the entire aggregate, the Document's Cover Texts may be placed on
covers that bracket the Document within the aggregate, or the
electronic equivalent of covers if the Document is in electronic form.
Otherwise they must appear on printed covers that bracket the whole
aggregate.


\begin{center}
{\Large\bfseries 8. TRANSLATION\par}
\end{center}
Translation is considered a kind of modification, so you may
distribute translations of the Document under the terms of section~4.
Replacing Invariant Sections with translations requires special
permission from their copyright holders, but you may include
translations of some or all Invariant Sections in addition to the
original versions of these Invariant Sections.  You may include a
translation of this License, and all the license notices in the
Document, and any Warranty Disclaimers, provided that you also include
the original English version of this License and the original versions
of those notices and disclaimers.  In case of a disagreement between
the translation and the original version of this License or a notice
or disclaimer, the original version will prevail.

If a section in the Document is Entitled ``Acknowledgements'',
``Dedications'', or ``History'', the requirement (section~4) to Preserve
its Title (section~1) will typically require changing the actual
title.


\begin{center}
{\Large\bfseries 9. TERMINATION\par}
\end{center}
You may not copy, modify, sublicense, or distribute the Document
except as expressly provided under this License.  Any attempt
otherwise to copy, modify, sublicense, or distribute it is void, and
will automatically terminate your rights under this License.

However, if you cease all violation of this License, then your license
from a particular copyright holder is reinstated (a) provisionally,
unless and until the copyright holder explicitly and finally
terminates your license, and (b) permanently, if the copyright holder
fails to notify you of the violation by some reasonable means prior to
60 days after the cessation.

Moreover, your license from a particular copyright holder is
reinstated permanently if the copyright holder notifies you of the
violation by some reasonable means, this is the first time you have
received notice of violation of this License (for any work) from that
copyright holder, and you cure the violation prior to 30 days after
your receipt of the notice.

Termination of your rights under this section does not terminate the
licenses of parties who have received copies or rights from you under
this License.  If your rights have been terminated and not permanently
reinstated, receipt of a copy of some or all of the same material does
not give you any rights to use it.


\begin{center}
{\Large\bfseries 10. FUTURE REVISIONS OF THIS LICENSE\par}
\end{center}
The Free Software Foundation may publish new, revised versions
of the GNU Free Documentation License from time to time.  Such new
versions will be similar in spirit to the present version, but may
differ in detail to address new problems or concerns.  See
\url{https://www.gnu.org/licenses/}.

Each version of the License is given a distinguishing version number.
If the Document specifies that a particular numbered version of this
License ``or any later version'' applies to it, you have the option of
following the terms and conditions either of that specified version or
of any later version that has been published (not as a draft) by the
Free Software Foundation.  If the Document does not specify a version
number of this License, you may choose any version ever published (not
as a draft) by the Free Software Foundation.  If the Document
specifies that a proxy can decide which future versions of this
License can be used, that proxy's public statement of acceptance of a
version permanently authorizes you to choose that version for the
Document.


\begin{center}
{\Large\bfseries 11. RELICENSING\par}
\end{center}
``Massive Multiauthor Collaboration Site'' (or ``MMC Site'') means any
World Wide Web server that publishes copyrightable works and also
provides prominent facilities for anybody to edit those works.  A
public wiki that anybody can edit is an example of such a server.  A
``Massive Multiauthor Collaboration'' (or ``MMC'') contained in the
site means any set of copyrightable works thus published on the MMC
site.

``CC-BY-SA'' means the Creative Commons Attribution-Share Alike 3.0
license published by Creative Commons Corporation, a not-for-profit
corporation with a principal place of business in San Francisco,
California, as well as future copyleft versions of that license
published by that same organization.

``Incorporate'' means to publish or republish a Document, in whole or
in part, as part of another Document.

An MMC is ``eligible for relicensing'' if it is licensed under this
License, and if all works that were first published under this License
somewhere other than this MMC, and subsequently incorporated in whole
or in part into the MMC, (1) had no cover texts or invariant sections,
and (2) were thus incorporated prior to November 1, 2008.

The operator of an MMC Site may republish an MMC contained in the site
under CC-BY-SA on the same site at any time before August 1, 2009,
provided the MMC is eligible for relicensing.


\begin{center}
{\Large\bfseries ADDENDUM: How to use this License for your documents\par}
\end{center}
To use this License in a document you have written, include a copy of
the License in the document and put the following copyright and
license notices just after the title page:

\bigskip
\begin{quote}
    Copyright \copyright{}  YEAR  YOUR NAME.
    Permission is granted to copy, distribute and/or modify this document
    under the terms of the GNU Free Documentation License, Version 1.3
    or any later version published by the Free Software Foundation;
    with no Invariant Sections, no Front-Cover Texts, and no Back-Cover Texts.
    A copy of the license is included in the section entitled ``GNU
    Free Documentation License''.
\end{quote}
\bigskip
    
If you have Invariant Sections, Front-Cover Texts and Back-Cover Texts,
replace the ``with \dots\ Texts.''\ line with this:

\bigskip
\begin{quote}
    with the Invariant Sections being LIST THEIR TITLES, with the
    Front-Cover Texts being LIST, and with the Back-Cover Texts being LIST.
\end{quote}
\bigskip
    
If you have Invariant Sections without Cover Texts, or some other
combination of the three, merge those two alternatives to suit the
situation.

If your document contains nontrivial examples of program code, we
recommend releasing these examples in parallel under your choice of
free software license, such as the GNU General Public License,
to permit their use in free software.

\chapter{GNU General Public License}\label{chp:gnugpl}

\begin{center}
{\parindent 0in
Version 3, 29 June 2007\\
Copyright \copyright\  2007 Free Software Foundation, Inc.\\
\url{https://fsf.org/}

\bigskip
Everyone is permitted to copy and distribute verbatim copies of this

license document, but changing it is not allowed.}

\end{center}

\begin{center}
{\Large\textsc{Preamble}}
\end{center}

The GNU General Public License is a free, copyleft license for
software and other kinds of works.

The licenses for most software and other practical works are designed
to take away your freedom to share and change the works.  By contrast,
the GNU General Public License is intended to guarantee your freedom to
share and change all versions of a program--to make sure it remains free
software for all its users.  We, the Free Software Foundation, use the
GNU General Public License for most of our software; it applies also to
any other work released this way by its authors.  You can apply it to
your programs, too.

When we speak of free software, we are referring to freedom, not
price.  Our General Public Licenses are designed to make sure that you
have the freedom to distribute copies of free software (and charge for
them if you wish), that you receive source code or can get it if you
want it, that you can change the software or use pieces of it in new
free programs, and that you know you can do these things.

To protect your rights, we need to prevent others from denying you
these rights or asking you to surrender the rights.  Therefore, you have
certain responsibilities if you distribute copies of the software, or if
you modify it: responsibilities to respect the freedom of others.

For example, if you distribute copies of such a program, whether
gratis or for a fee, you must pass on to the recipients the same
freedoms that you received.  You must make sure that they, too, receive
or can get the source code.  And you must show them these terms so they
know their rights.

Developers that use the GNU GPL protect your rights with two steps:
(1) assert copyright on the software, and (2) offer you this License
giving you legal permission to copy, distribute and/or modify it.

For the developers' and authors' protection, the GPL clearly explains
that there is no warranty for this free software.  For both users' and
authors' sake, the GPL requires that modified versions be marked as
changed, so that their problems will not be attributed erroneously to
authors of previous versions.

Some devices are designed to deny users access to install or run
modified versions of the software inside them, although the manufacturer
can do so.  This is fundamentally incompatible with the aim of
protecting users' freedom to change the software.  The systematic
pattern of such abuse occurs in the area of products for individuals to
use, which is precisely where it is most unacceptable.  Therefore, we
have designed this version of the GPL to prohibit the practice for those
products.  If such problems arise substantially in other domains, we
stand ready to extend this provision to those domains in future versions
of the GPL, as needed to protect the freedom of users.

Finally, every program is threatened constantly by software patents.
States should not allow patents to restrict development and use of
software on general-purpose computers, but in those that do, we wish to
avoid the special danger that patents applied to a free program could
make it effectively proprietary.  To prevent this, the GPL assures that
patents cannot be used to render the program non-free.

The precise terms and conditions for copying, distribution and
modification follow.

\begin{center}
{\Large\textsc{Terms and Conditions}}
\end{center}


\begin{enumerate}

\addtocounter{enumi}{-1}

\item Definitions.

``This License'' refers to version 3 of the GNU General Public License.

``Copyright'' also means copyright-like laws that apply to other kinds of
works, such as semiconductor masks.

``The Program'' refers to any copyrightable work licensed under this
License.  Each licensee is addressed as ``you''.  ``Licensees'' and
``recipients'' may be individuals or organizations.

To ``modify'' a work means to copy from or adapt all or part of the work
in a fashion requiring copyright permission, other than the making of an
exact copy.  The resulting work is called a ``modified version'' of the
earlier work or a work ``based on'' the earlier work.

A ``covered work'' means either the unmodified Program or a work based
on the Program.

To ``propagate'' a work means to do anything with it that, without
permission, would make you directly or secondarily liable for
infringement under applicable copyright law, except executing it on a
computer or modifying a private copy.  Propagation includes copying,
distribution (with or without modification), making available to the
public, and in some countries other activities as well.

To ``convey'' a work means any kind of propagation that enables other
parties to make or receive copies.  Mere interaction with a user through
a computer network, with no transfer of a copy, is not conveying.

An interactive user interface displays ``Appropriate Legal Notices''
to the extent that it includes a convenient and prominently visible
feature that (1) displays an appropriate copyright notice, and (2)
tells the user that there is no warranty for the work (except to the
extent that warranties are provided), that licensees may convey the
work under this License, and how to view a copy of this License.  If
the interface presents a list of user commands or options, such as a
menu, a prominent item in the list meets this criterion.

\item Source Code.

The ``source code'' for a work means the preferred form of the work
for making modifications to it.  ``Object code'' means any non-source
form of a work.

A ``Standard Interface'' means an interface that either is an official
standard defined by a recognized standards body, or, in the case of
interfaces specified for a particular programming language, one that
is widely used among developers working in that language.

The ``System Libraries'' of an executable work include anything, other
than the work as a whole, that (a) is included in the normal form of
packaging a Major Component, but which is not part of that Major
Component, and (b) serves only to enable use of the work with that
Major Component, or to implement a Standard Interface for which an
implementation is available to the public in source code form.  A
``Major Component'', in this context, means a major essential component
(kernel, window system, and so on) of the specific operating system
(if any) on which the executable work runs, or a compiler used to
produce the work, or an object code interpreter used to run it.

The ``Corresponding Source'' for a work in object code form means all
the source code needed to generate, install, and (for an executable
work) run the object code and to modify the work, including scripts to
control those activities.  However, it does not include the work's
System Libraries, or general-purpose tools or generally available free
programs which are used unmodified in performing those activities but
which are not part of the work.  For example, Corresponding Source
includes interface definition files associated with source files for
the work, and the source code for shared libraries and dynamically
linked subprograms that the work is specifically designed to require,
such as by intimate data communication or control flow between those
subprograms and other parts of the work.

The Corresponding Source need not include anything that users
can regenerate automatically from other parts of the Corresponding
Source.

The Corresponding Source for a work in source code form is that
same work.

\item Basic Permissions.

All rights granted under this License are granted for the term of
copyright on the Program, and are irrevocable provided the stated
conditions are met.  This License explicitly affirms your unlimited
permission to run the unmodified Program.  The output from running a
covered work is covered by this License only if the output, given its
content, constitutes a covered work.  This License acknowledges your
rights of fair use or other equivalent, as provided by copyright law.

You may make, run and propagate covered works that you do not
convey, without conditions so long as your license otherwise remains
in force.  You may convey covered works to others for the sole purpose
of having them make modifications exclusively for you, or provide you
with facilities for running those works, provided that you comply with
the terms of this License in conveying all material for which you do
not control copyright.  Those thus making or running the covered works
for you must do so exclusively on your behalf, under your direction
and control, on terms that prohibit them from making any copies of
your copyrighted material outside their relationship with you.

Conveying under any other circumstances is permitted solely under
the conditions stated below.  Sublicensing is not allowed; section 10
makes it unnecessary.

\item Protecting Users' Legal Rights From Anti-Circumvention Law.

No covered work shall be deemed part of an effective technological
measure under any applicable law fulfilling obligations under article
11 of the WIPO copyright treaty adopted on 20 December 1996, or
similar laws prohibiting or restricting circumvention of such
measures.

When you convey a covered work, you waive any legal power to forbid
circumvention of technological measures to the extent such circumvention
is effected by exercising rights under this License with respect to
the covered work, and you disclaim any intention to limit operation or
modification of the work as a means of enforcing, against the work's
users, your or third parties' legal rights to forbid circumvention of
technological measures.

\item Conveying Verbatim Copies.

You may convey verbatim copies of the Program's source code as you
receive it, in any medium, provided that you conspicuously and
appropriately publish on each copy an appropriate copyright notice;
keep intact all notices stating that this License and any
non-permissive terms added in accord with section 7 apply to the code;
keep intact all notices of the absence of any warranty; and give all
recipients a copy of this License along with the Program.

You may charge any price or no price for each copy that you convey,
and you may offer support or warranty protection for a fee.

\item Conveying Modified Source Versions.

You may convey a work based on the Program, or the modifications to
produce it from the Program, in the form of source code under the
terms of section 4, provided that you also meet all of these conditions:
  \begin{enumerate}
  \item The work must carry prominent notices stating that you modified
  it, and giving a relevant date.

  \item The work must carry prominent notices stating that it is
  released under this License and any conditions added under section
  7.  This requirement modifies the requirement in section 4 to
  ``keep intact all notices''.

  \item You must license the entire work, as a whole, under this
  License to anyone who comes into possession of a copy.  This
  License will therefore apply, along with any applicable section 7
  additional terms, to the whole of the work, and all its parts,
  regardless of how they are packaged.  This License gives no
  permission to license the work in any other way, but it does not
  invalidate such permission if you have separately received it.

  \item If the work has interactive user interfaces, each must display
  Appropriate Legal Notices; however, if the Program has interactive
  interfaces that do not display Appropriate Legal Notices, your
  work need not make them do so.
\end{enumerate}
A compilation of a covered work with other separate and independent
works, which are not by their nature extensions of the covered work,
and which are not combined with it such as to form a larger program,
in or on a volume of a storage or distribution medium, is called an
``aggregate'' if the compilation and its resulting copyright are not
used to limit the access or legal rights of the compilation's users
beyond what the individual works permit.  Inclusion of a covered work
in an aggregate does not cause this License to apply to the other
parts of the aggregate.

\item Conveying Non-Source Forms.

You may convey a covered work in object code form under the terms
of sections 4 and 5, provided that you also convey the
machine-readable Corresponding Source under the terms of this License,
in one of these ways:
  \begin{enumerate}
  \item Convey the object code in, or embodied in, a physical product
  (including a physical distribution medium), accompanied by the
  Corresponding Source fixed on a durable physical medium
  customarily used for software interchange.

  \item Convey the object code in, or embodied in, a physical product
  (including a physical distribution medium), accompanied by a
  written offer, valid for at least three years and valid for as
  long as you offer spare parts or customer support for that product
  model, to give anyone who possesses the object code either (1) a
  copy of the Corresponding Source for all the software in the
  product that is covered by this License, on a durable physical
  medium customarily used for software interchange, for a price no
  more than your reasonable cost of physically performing this
  conveying of source, or (2) access to copy the
  Corresponding Source from a network server at no charge.

  \item Convey individual copies of the object code with a copy of the
  written offer to provide the Corresponding Source.  This
  alternative is allowed only occasionally and noncommercially, and
  only if you received the object code with such an offer, in accord
  with subsection 6b.

  \item Convey the object code by offering access from a designated
  place (gratis or for a charge), and offer equivalent access to the
  Corresponding Source in the same way through the same place at no
  further charge.  You need not require recipients to copy the
  Corresponding Source along with the object code.  If the place to
  copy the object code is a network server, the Corresponding Source
  may be on a different server (operated by you or a third party)
  that supports equivalent copying facilities, provided you maintain
  clear directions next to the object code saying where to find the
  Corresponding Source.  Regardless of what server hosts the
  Corresponding Source, you remain obligated to ensure that it is
  available for as long as needed to satisfy these requirements.

  \item Convey the object code using peer-to-peer transmission, provided
  you inform other peers where the object code and Corresponding
  Source of the work are being offered to the general public at no
  charge under subsection 6d.
  \end{enumerate}

A separable portion of the object code, whose source code is excluded
from the Corresponding Source as a System Library, need not be
included in conveying the object code work.

A ``User Product'' is either (1) a ``consumer product'', which means any
tangible personal property which is normally used for personal, family,
or household purposes, or (2) anything designed or sold for incorporation
into a dwelling.  In determining whether a product is a consumer product,
doubtful cases shall be resolved in favor of coverage.  For a particular
product received by a particular user, ``normally used'' refers to a
typical or common use of that class of product, regardless of the status
of the particular user or of the way in which the particular user
actually uses, or expects or is expected to use, the product.  A product
is a consumer product regardless of whether the product has substantial
commercial, industrial or non-consumer uses, unless such uses represent
the only significant mode of use of the product.

``Installation Information'' for a User Product means any methods,
procedures, authorization keys, or other information required to install
and execute modified versions of a covered work in that User Product from
a modified version of its Corresponding Source.  The information must
suffice to ensure that the continued functioning of the modified object
code is in no case prevented or interfered with solely because
modification has been made.

If you convey an object code work under this section in, or with, or
specifically for use in, a User Product, and the conveying occurs as
part of a transaction in which the right of possession and use of the
User Product is transferred to the recipient in perpetuity or for a
fixed term (regardless of how the transaction is characterized), the
Corresponding Source conveyed under this section must be accompanied
by the Installation Information.  But this requirement does not apply
if neither you nor any third party retains the ability to install
modified object code on the User Product (for example, the work has
been installed in ROM).

The requirement to provide Installation Information does not include a
requirement to continue to provide support service, warranty, or updates
for a work that has been modified or installed by the recipient, or for
the User Product in which it has been modified or installed.  Access to a
network may be denied when the modification itself materially and
adversely affects the operation of the network or violates the rules and
protocols for communication across the network.

Corresponding Source conveyed, and Installation Information provided,
in accord with this section must be in a format that is publicly
documented (and with an implementation available to the public in
source code form), and must require no special password or key for
unpacking, reading or copying.

\item Additional Terms.

``Additional permissions'' are terms that supplement the terms of this
License by making exceptions from one or more of its conditions.
Additional permissions that are applicable to the entire Program shall
be treated as though they were included in this License, to the extent
that they are valid under applicable law.  If additional permissions
apply only to part of the Program, that part may be used separately
under those permissions, but the entire Program remains governed by
this License without regard to the additional permissions.

When you convey a copy of a covered work, you may at your option
remove any additional permissions from that copy, or from any part of
it.  (Additional permissions may be written to require their own
removal in certain cases when you modify the work.)  You may place
additional permissions on material, added by you to a covered work,
for which you have or can give appropriate copyright permission.

Notwithstanding any other provision of this License, for material you
add to a covered work, you may (if authorized by the copyright holders of
that material) supplement the terms of this License with terms:
  \begin{enumerate}
  \item Disclaiming warranty or limiting liability differently from the
  terms of sections 15 and 16 of this License; or

  \item Requiring preservation of specified reasonable legal notices or
  author attributions in that material or in the Appropriate Legal
  Notices displayed by works containing it; or

  \item Prohibiting misrepresentation of the origin of that material, or
  requiring that modified versions of such material be marked in
  reasonable ways as different from the original version; or

  \item Limiting the use for publicity purposes of names of licensors or
  authors of the material; or

  \item Declining to grant rights under trademark law for use of some
  trade names, trademarks, or service marks; or

  \item Requiring indemnification of licensors and authors of that
  material by anyone who conveys the material (or modified versions of
  it) with contractual assumptions of liability to the recipient, for
  any liability that these contractual assumptions directly impose on
  those licensors and authors.
  \end{enumerate}

All other non-permissive additional terms are considered ``further
restrictions'' within the meaning of section 10.  If the Program as you
received it, or any part of it, contains a notice stating that it is
governed by this License along with a term that is a further
restriction, you may remove that term.  If a license document contains
a further restriction but permits relicensing or conveying under this
License, you may add to a covered work material governed by the terms
of that license document, provided that the further restriction does
not survive such relicensing or conveying.

If you add terms to a covered work in accord with this section, you
must place, in the relevant source files, a statement of the
additional terms that apply to those files, or a notice indicating
where to find the applicable terms.

Additional terms, permissive or non-permissive, may be stated in the
form of a separately written license, or stated as exceptions;
the above requirements apply either way.

\item Termination.

You may not propagate or modify a covered work except as expressly
provided under this License.  Any attempt otherwise to propagate or
modify it is void, and will automatically terminate your rights under
this License (including any patent licenses granted under the third
paragraph of section 11).

However, if you cease all violation of this License, then your
license from a particular copyright holder is reinstated (a)
provisionally, unless and until the copyright holder explicitly and
finally terminates your license, and (b) permanently, if the copyright
holder fails to notify you of the violation by some reasonable means
prior to 60 days after the cessation.

Moreover, your license from a particular copyright holder is
reinstated permanently if the copyright holder notifies you of the
violation by some reasonable means, this is the first time you have
received notice of violation of this License (for any work) from that
copyright holder, and you cure the violation prior to 30 days after
your receipt of the notice.

Termination of your rights under this section does not terminate the
licenses of parties who have received copies or rights from you under
this License.  If your rights have been terminated and not permanently
reinstated, you do not qualify to receive new licenses for the same
material under section 10.

\item Acceptance Not Required for Having Copies.

You are not required to accept this License in order to receive or
run a copy of the Program.  Ancillary propagation of a covered work
occurring solely as a consequence of using peer-to-peer transmission
to receive a copy likewise does not require acceptance.  However,
nothing other than this License grants you permission to propagate or
modify any covered work.  These actions infringe copyright if you do
not accept this License.  Therefore, by modifying or propagating a
covered work, you indicate your acceptance of this License to do so.

\item Automatic Licensing of Downstream Recipients.

Each time you convey a covered work, the recipient automatically
receives a license from the original licensors, to run, modify and
propagate that work, subject to this License.  You are not responsible
for enforcing compliance by third parties with this License.

An ``entity transaction'' is a transaction transferring control of an
organization, or substantially all assets of one, or subdividing an
organization, or merging organizations.  If propagation of a covered
work results from an entity transaction, each party to that
transaction who receives a copy of the work also receives whatever
licenses to the work the party's predecessor in interest had or could
give under the previous paragraph, plus a right to possession of the
Corresponding Source of the work from the predecessor in interest, if
the predecessor has it or can get it with reasonable efforts.

You may not impose any further restrictions on the exercise of the
rights granted or affirmed under this License.  For example, you may
not impose a license fee, royalty, or other charge for exercise of
rights granted under this License, and you may not initiate litigation
(including a cross-claim or counterclaim in a lawsuit) alleging that
any patent claim is infringed by making, using, selling, offering for
sale, or importing the Program or any portion of it.

\item Patents.

A ``contributor'' is a copyright holder who authorizes use under this
License of the Program or a work on which the Program is based.  The
work thus licensed is called the contributor's ``contributor version''.

A contributor's ``essential patent claims'' are all patent claims
owned or controlled by the contributor, whether already acquired or
hereafter acquired, that would be infringed by some manner, permitted
by this License, of making, using, or selling its contributor version,
but do not include claims that would be infringed only as a
consequence of further modification of the contributor version.  For
purposes of this definition, ``control'' includes the right to grant
patent sublicenses in a manner consistent with the requirements of
this License.

Each contributor grants you a non-exclusive, worldwide, royalty-free
patent license under the contributor's essential patent claims, to
make, use, sell, offer for sale, import and otherwise run, modify and
propagate the contents of its contributor version.

In the following three paragraphs, a ``patent license'' is any express
agreement or commitment, however denominated, not to enforce a patent
(such as an express permission to practice a patent or covenant not to
sue for patent infringement).  To ``grant'' such a patent license to a
party means to make such an agreement or commitment not to enforce a
patent against the party.

If you convey a covered work, knowingly relying on a patent license,
and the Corresponding Source of the work is not available for anyone
to copy, free of charge and under the terms of this License, through a
publicly available network server or other readily accessible means,
then you must either (1) cause the Corresponding Source to be so
available, or (2) arrange to deprive yourself of the benefit of the
patent license for this particular work, or (3) arrange, in a manner
consistent with the requirements of this License, to extend the patent
license to downstream recipients.  ``Knowingly relying'' means you have
actual knowledge that, but for the patent license, your conveying the
covered work in a country, or your recipient's use of the covered work
in a country, would infringe one or more identifiable patents in that
country that you have reason to believe are valid.

If, pursuant to or in connection with a single transaction or
arrangement, you convey, or propagate by procuring conveyance of, a
covered work, and grant a patent license to some of the parties
receiving the covered work authorizing them to use, propagate, modify
or convey a specific copy of the covered work, then the patent license
you grant is automatically extended to all recipients of the covered
work and works based on it.

A patent license is ``discriminatory'' if it does not include within
the scope of its coverage, prohibits the exercise of, or is
conditioned on the non-exercise of one or more of the rights that are
specifically granted under this License.  You may not convey a covered
work if you are a party to an arrangement with a third party that is
in the business of distributing software, under which you make payment
to the third party based on the extent of your activity of conveying
the work, and under which the third party grants, to any of the
parties who would receive the covered work from you, a discriminatory
patent license (a) in connection with copies of the covered work
conveyed by you (or copies made from those copies), or (b) primarily
for and in connection with specific products or compilations that
contain the covered work, unless you entered into that arrangement,
or that patent license was granted, prior to 28 March 2007.

Nothing in this License shall be construed as excluding or limiting
any implied license or other defenses to infringement that may
otherwise be available to you under applicable patent law.

\item No Surrender of Others' Freedom.

If conditions are imposed on you (whether by court order, agreement or
otherwise) that contradict the conditions of this License, they do not
excuse you from the conditions of this License.  If you cannot convey a
covered work so as to satisfy simultaneously your obligations under this
License and any other pertinent obligations, then as a consequence you may
not convey it at all.  For example, if you agree to terms that obligate you
to collect a royalty for further conveying from those to whom you convey
the Program, the only way you could satisfy both those terms and this
License would be to refrain entirely from conveying the Program.

\item Use with the GNU Affero General Public License.

Notwithstanding any other provision of this License, you have
permission to link or combine any covered work with a work licensed
under version 3 of the GNU Affero General Public License into a single
combined work, and to convey the resulting work.  The terms of this
License will continue to apply to the part which is the covered work,
but the special requirements of the GNU Affero General Public License,
section 13, concerning interaction through a network will apply to the
combination as such.

\item Revised Versions of this License.

The Free Software Foundation may publish revised and/or new versions of
the GNU General Public License from time to time.  Such new versions will
be similar in spirit to the present version, but may differ in detail to
address new problems or concerns.

Each version is given a distinguishing version number.  If the
Program specifies that a certain numbered version of the GNU General
Public License ``or any later version'' applies to it, you have the
option of following the terms and conditions either of that numbered
version or of any later version published by the Free Software
Foundation.  If the Program does not specify a version number of the
GNU General Public License, you may choose any version ever published
by the Free Software Foundation.

If the Program specifies that a proxy can decide which future
versions of the GNU General Public License can be used, that proxy's
public statement of acceptance of a version permanently authorizes you
to choose that version for the Program.

Later license versions may give you additional or different
permissions.  However, no additional obligations are imposed on any
author or copyright holder as a result of your choosing to follow a
later version.

\item Disclaimer of Warranty.

\begin{sloppypar}
 THERE IS NO WARRANTY FOR THE PROGRAM, TO THE EXTENT PERMITTED BY
 APPLICABLE LAW.  EXCEPT WHEN OTHERWISE STATED IN WRITING THE
 COPYRIGHT HOLDERS AND/OR OTHER PARTIES PROVIDE THE PROGRAM ``AS IS''
 WITHOUT WARRANTY OF ANY KIND, EITHER EXPRESSED OR IMPLIED,
 INCLUDING, BUT NOT LIMITED TO, THE IMPLIED WARRANTIES OF
 MERCHANTABILITY AND FITNESS FOR A PARTICULAR PURPOSE.  THE ENTIRE
 RISK AS TO THE QUALITY AND PERFORMANCE OF THE PROGRAM IS WITH YOU.
 SHOULD THE PROGRAM PROVE DEFECTIVE, YOU ASSUME THE COST OF ALL
 NECESSARY SERVICING, REPAIR OR CORRECTION.
\end{sloppypar}

\item Limitation of Liability.

 IN NO EVENT UNLESS REQUIRED BY APPLICABLE LAW OR AGREED TO IN
 WRITING WILL ANY COPYRIGHT HOLDER, OR ANY OTHER PARTY WHO MODIFIES
 AND/OR CONVEYS THE PROGRAM AS PERMITTED ABOVE, BE LIABLE TO YOU FOR
 DAMAGES, INCLUDING ANY GENERAL, SPECIAL, INCIDENTAL OR CONSEQUENTIAL
 DAMAGES ARISING OUT OF THE USE OR INABILITY TO USE THE PROGRAM
 (INCLUDING BUT NOT LIMITED TO LOSS OF DATA OR DATA BEING RENDERED
 INACCURATE OR LOSSES SUSTAINED BY YOU OR THIRD PARTIES OR A FAILURE
 OF THE PROGRAM TO OPERATE WITH ANY OTHER PROGRAMS), EVEN IF SUCH
 HOLDER OR OTHER PARTY HAS BEEN ADVISED OF THE POSSIBILITY OF SUCH
 DAMAGES.

\item Interpretation of Sections 15 and 16.

If the disclaimer of warranty and limitation of liability provided
above cannot be given local legal effect according to their terms,
reviewing courts shall apply local law that most closely approximates
an absolute waiver of all civil liability in connection with the
Program, unless a warranty or assumption of liability accompanies a
copy of the Program in return for a fee.

\begin{center}
{\Large\textsc{End of Terms and Conditions}}

\bigskip
How to Apply These Terms to Your New Programs
\end{center}

If you develop a new program, and you want it to be of the greatest
possible use to the public, the best way to achieve this is to make it
free software which everyone can redistribute and change under these terms.

To do so, attach the following notices to the program.  It is safest
to attach them to the start of each source file to most effectively
state the exclusion of warranty; and each file should have at least
the ``copyright'' line and a pointer to where the full notice is found.

{\footnotesize
\begin{verbatim}
<one line to give the program's name and a brief idea of what it does.>

Copyright (C) <textyear>  <name of author>

This program is free software: you can redistribute it and/or modify
it under the terms of the GNU General Public License as published by
the Free Software Foundation, either version 3 of the License, or
(at your option) any later version.

This program is distributed in the hope that it will be useful,
but WITHOUT ANY WARRANTY; without even the implied warranty of
MERCHANTABILITY or FITNESS FOR A PARTICULAR PURPOSE.  See the
GNU General Public License for more details.

You should have received a copy of the GNU General Public License
along with this program.  If not, see <https://www.gnu.org/licenses/>.
\end{verbatim}
}

Also add information on how to contact you by electronic and paper mail.

If the program does terminal interaction, make it output a short
notice like this when it starts in an interactive mode:

{\footnotesize
\begin{verbatim}
<program>  Copyright (C) <year>  <name of author>

This program comes with ABSOLUTELY NO WARRANTY; for details type `show w'.
This is free software, and you are welcome to redistribute it
under certain conditions; type `show c' for details.
\end{verbatim}
}

The hypothetical commands \texttt{show w} and \texttt{show c} should show
the appropriate parts of the General Public License.  Of course, your program's
commands might be different; for a GUI interface, you would use an
``about box''.

You should also get your employer (if you work as a programmer) or
school, if any, to sign a ``copyright disclaimer'' for the program, if
necessary.  For more information on this, and how to apply and follow
the GNU GPL, see \url{https://www.gnu.org/licenses/}.

The GNU General Public License does not permit incorporating your
program into proprietary programs.  If your program is a subroutine
library, you may consider it more useful to permit linking proprietary
applications with the library.  If this is what you want to do, use
the GNU Lesser General Public License instead of this License.  But
first, please read \url{https://www.gnu.org/licenses/why-not-lgpl.html}.

\end{enumerate}




\backmatter
\printbibliography

\end{document}
